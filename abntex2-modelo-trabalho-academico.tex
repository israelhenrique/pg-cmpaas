%% abtex2-modelo-trabalho-academico.tex, v-1.9.6 laurocesar
%% Copyright 2012-2016 by abnTeX2 group at http://www.abntex.net.br/ 
%%
%% This work may be distributed and/or modified under the
%% conditions of the LaTeX Project Public License, either version 1.3
%% of this license or (at your option) any later version.
%% The latest version of this license is in
%%   http://www.latex-project.org/lppl.txt
%% and version 1.3 or later is part of all distributions of LaTeX
%% version 2005/12/01 or later.
%%
%% This work has the LPPL maintenance status `maintained'.
%% 
%% The Current Maintainer of this work is the abnTeX2 team, led
%% by Lauro César Araujo. Further information are available on 
%% http://www.abntex.net.br/
%%
%% This work consists of the files abntex2-modelo-trabalho-academico.tex,
%% abntex2-modelo-include-comandos and abntex2-modelo-references.bib
%%

% ------------------------------------------------------------------------
% ------------------------------------------------------------------------
% abnTeX2: Modelo de Trabalho Academico (tese de doutorado, dissertacao de
% mestrado e trabalhos monograficos em geral) em conformidade com 
% ABNT NBR 14724:2011: Informacao e documentacao - Trabalhos academicos -
% Apresentacao
% ------------------------------------------------------------------------
% ------------------------------------------------------------------------

\documentclass[
	% -- opções da classe memoir --
	12pt,				% tamanho da fonte
	openright,			% capítulos começam em pág ímpar (insere página vazia caso preciso)
	oneside,			% para impressão em recto e verso. Oposto a oneside
	a4paper,			% tamanho do papel. 
	% -- opções da classe abntex2 --
	%chapter=TITLE,		% títulos de capítulos convertidos em letras maiúsculas
	%section=TITLE,		% títulos de seções convertidos em letras maiúsculas
	%subsection=TITLE,	% títulos de subseções convertidos em letras maiúsculas
	%subsubsection=TITLE,% títulos de subsubseções convertidos em letras maiúsculas
	% -- opções do pacote babel --
	english,			% idioma adicional para hifenização
	french,				% idioma adicional para hifenização
	spanish,			% idioma adicional para hifenização
	brazil				% o último idioma é o principal do documento
	]{abntex2}

% ---
% Pacotes básicos 
% ---
\usepackage{lmodern}			% Usa a fonte Latin Modern			
\usepackage[T1]{fontenc}		% Selecao de codigos de fonte.
\usepackage[utf8]{inputenc}		% Codificacao do documento (conversão automática dos acentos)
\usepackage{lastpage}			% Usado pela Ficha catalográfica
\usepackage{indentfirst}		% Indenta o primeiro parágrafo de cada seção.
\usepackage{color}				% Controle das cores
\usepackage{graphicx}			% Inclusão de gráficos
\usepackage{microtype} 			% para melhorias de justificação
% ---
		
% ---
% Pacotes adicionais, usados apenas no âmbito do Modelo Canônico do abnteX2
% ---
\usepackage{lipsum}				% para geração de dummy text
% ---

% ---
% Pacotes de citações
% ---
%\usepackage[brazilian,hyperpageref]{backref}	 % Paginas com as citações na bibl
\usepackage[alf]{abntex2cite}	% Citações padrão ABNT

% --- 
% CONFIGURAÇÕES DE PACOTES
% --- 

% ---
% Configurações do pacote backref
% Usado sem a opção hyperpageref de backref
%\renewcommand{\backrefpagesname}{Citado na(s) página(s):~}
% Texto padrão antes do número das páginas
%\renewcommand{\backref}{}
% Define os textos da citação
%\renewcommand*{\backrefalt}[4]{
%	\ifcase #1 %
%		Nenhuma citação no texto.%
%	\or
%		Citado na página #2.%
%	\else
%		Citado #1 vezes nas páginas #2.%
%	\fi}%
% ---

% ---
% Informações de dados para CAPA e FOLHA DE ROSTO
% ---
\titulo{Um plugin de mapas conceituais para o Moodle}
\autor{Israel Henrique Silva de Lima}
\local{Vitória}
\data{2016}
\orientador{Wagner de Andrade  Perin}
\coorientador{Davidson Cury}
\instituicao{
	Universidade Federal do Espírito Santo -- UFES
	\par
	Centro Tecnológico
	\par
	Departamento de Informática}
\tipotrabalho{Monografia (PG)}
% O preambulo deve conter o tipo do trabalho, o objetivo, 
% o nome da instituição e a área de concentração 
\preambulo{Monografia apresentada ao Curso de Engenharia de Computação do Departamento de Informática da Universidade Federal do Espírito Santo, como requisito parcial para obtenção do Grau de Bacharel em Engenharia de Computação.}
% ---


% ---
% Configurações de aparência do PDF final

% alterando o aspecto da cor azul
\definecolor{blue}{RGB}{41,5,195}

% informações do PDF
\makeatletter
\hypersetup{
     	%pagebackref=true,
		pdftitle={\@title}, 
		pdfauthor={\@author},
    	pdfsubject={\imprimirpreambulo},
	    pdfcreator={LaTeX with abnTeX2},
		pdfkeywords={abnt}{latex}{abntex}{abntex2}{trabalho acadêmico}, 
		colorlinks=true,       		% false: boxed links; true: colored links
    	linkcolor=blue,          	% color of internal links
    	citecolor=blue,        		% color of links to bibliography
    	filecolor=magenta,      		% color of file links
		urlcolor=blue,
		bookmarksdepth=4
}
\makeatother
% --- 

% --- 
% Espaçamentos entre linhas e parágrafos 
% --- 

% O tamanho do parágrafo é dado por:
\setlength{\parindent}{1.3cm}

% Controle do espaçamento entre um parágrafo e outro:
\setlength{\parskip}{0.2cm}  % tente também \onelineskip

% ---
% compila o indice
% ---
\makeindex
% ---

% ----
% Início do documento
% ----
\begin{document}

% Seleciona o idioma do documento (conforme pacotes do babel)
%\selectlanguage{english}
\selectlanguage{brazil}

% Retira espaço extra obsoleto entre as frases.
\frenchspacing 

% ----------------------------------------------------------
% ELEMENTOS PRÉ-TEXTUAIS
% ----------------------------------------------------------
% \pretextual

% ---
% Capa
% ---
\imprimircapa
% ---

% ---
% Folha de rosto
% (o * indica que haverá a ficha bibliográfica)
% ---
\imprimirfolhaderosto*
% ---

% ---
% Inserir a ficha bibliografica
% ---

% Isto é um exemplo de Ficha Catalográfica, ou ``Dados internacionais de
% catalogação-na-publicação''. Você pode utilizar este modelo como referência. 
% Porém, provavelmente a biblioteca da sua universidade lhe fornecerá um PDF
% com a ficha catalográfica definitiva após a defesa do trabalho. Quando estiver
% com o documento, salve-o como PDF no diretório do seu projeto e substitua todo
% o conteúdo de implementação deste arquivo pelo comando abaixo:
%
% \begin{fichacatalografica}
%     \includepdf{fig_ficha_catalografica.pdf}
% \end{fichacatalografica}

%\begin{fichacatalografica}
%	\sffamily
%	\vspace*{\fill}					% Posição vertical
%	\begin{center}					% Minipage Centralizado
%	\fbox{\begin{minipage}[c][8cm]{13.5cm}		% Largura
%	\small
%	\imprimirautor
%	%Sobrenome, Nome do autor
%	
%	\hspace{0.5cm} \imprimirtitulo  / \imprimirautor. --
%	\imprimirlocal, \imprimirdata-
%	
%	\hspace{0.5cm} \pageref{LastPage} p. : il. (algumas color.) ; 30 cm.\\
%	
%	\hspace{0.5cm} \imprimirorientadorRotulo~\imprimirorientador\\
%	
%	\hspace{0.5cm}
%	\parbox[t]{\textwidth}{\imprimirtipotrabalho~--~\imprimirinstituicao,
%	\imprimirdata.}\\
%	
%	\hspace{0.5cm}
%		1. Palavra-chave1.
%		2. Palavra-chave2.
%		2. Palavra-chave3.
%		II. Universidade Federal do Espírito Santo.
%		III. Faculdade de xxx.
%		IV. \imprimirtitulo 			
%	\end{minipage}}
%	\end{center}
%\end{fichacatalografica}
% ---



% ---
% Inserir folha de aprovação
% ---

% Isto é um exemplo de Folha de aprovação, elemento obrigatório da NBR
% 14724/2011 (seção 4.2.1.3). Você pode utilizar este modelo até a aprovação
% do trabalho. Após isso, substitua todo o conteúdo deste arquivo por uma
% imagem da página assinada pela banca com o comando abaixo:
%
% \includepdf{folhadeaprovacao_final.pdf}
%
\begin{folhadeaprovacao}

  \begin{center}
    {\ABNTEXchapterfont\large\imprimirautor}

    \vspace*{\fill}\vspace*{\fill}
    \begin{center}
      \ABNTEXchapterfont\bfseries\Large\imprimirtitulo
    \end{center}
    \vspace*{\fill}
    
    \hspace{.45\textwidth}
    \begin{minipage}{.5\textwidth}
        \imprimirpreambulo
    \end{minipage}%
    \vspace*{\fill}
   \end{center}
        
   Trabalho aprovado. \imprimirlocal, 24 de novembro de 2012:

   \assinatura{\textbf{\imprimirorientador} \\ Orientador} 
   \assinatura{\textbf{Professor} \\ Convidado 1}
   \assinatura{\textbf{Professor} \\ Convidado 2}
   %\assinatura{\textbf{Professor} \\ Convidado 3}
   %\assinatura{\textbf{Professor} \\ Convidado 4}
      
   \begin{center}
    \vspace*{0.5cm}
    {\large\imprimirlocal}
    \par
    {\large\imprimirdata}
    \vspace*{1cm}
  \end{center}
  
\end{folhadeaprovacao}
% ---

% ---
% Dedicatória
% ---
% \begin{dedicatoria}
%    \vspace*{\fill}
%    \centering
%    \noindent
%    \textit{ Este trabalho é dedicado às crianças adultas que,\\
%    quando pequenas, sonharam em se tornar cientistas.} \vspace*{\fill}
% \end{dedicatoria}
% ---

% ---
% Agradecimentos
% ---
% \begin{agradecimentos}
% \end{agradecimentos}
% ---

% ---
% Epígrafe
% ---
%\begin{epigrafe}
%     \vspace*{\fill}
% 	\begin{flushright}
% 		\textit{``Não vos amoldeis às estruturas deste mundo, \\
% 		mas transformai-vos pela renovação da mente, \\
% 		a fim de distinguir qual é a vontade de Deus: \\
% 		o que é bom, o que Lhe é agradável, o que é perfeito.\\
% 		(Bíblia Sagrada, Romanos 12, 2)}
% 	\end{flushright}
% \end{epigrafe}
% ---

% ---
% RESUMOS
% ---

% resumo em português
\setlength{\absparsep}{18pt} % ajusta o espaçamento dos parágrafos do resumo
\begin{resumo}
 Segundo a \citeonline[3.1-3.2]{NBR6028:2003}, o resumo deve ressaltar o
 objetivo, o método, os resultados e as conclusões do documento. A ordem e a extensão
 destes itens dependem do tipo de resumo (informativo ou indicativo) e do
 tratamento que cada item recebe no documento original. O resumo deve ser
 precedido da referência do documento, com exceção do resumo inserido no
 próprio documento. (\ldots) As palavras-chave devem figurar logo abaixo do
 resumo, antecedidas da expressão Palavras-chave:, separadas entre si por
 ponto e finalizadas também por ponto.

 \textbf{Palavras-chave}: latex. abntex. editoração de texto.
\end{resumo}

% resumo em inglês
% \begin{resumo}[Abstract]
%  \begin{otherlanguage*}{english}
%    This is the english abstract.
% 
%    \vspace{\onelineskip}
% 
%    \noindent 
%    \textbf{Keywords}: latex. abntex. text editoration.
%  \end{otherlanguage*}
% \end{resumo}


% ---
% inserir lista de ilustrações
% ---
\pdfbookmark[0]{\listfigurename}{lof}
\listoffigures*
\cleardoublepage
% ---

% ---
% inserir lista de tabelas
% ---
%\pdfbookmark[0]{\listtablename}{lot}
%\listoftables*
%\cleardoublepage
% ---

% ---
% inserir lista de abreviaturas e siglas
% ---
%\begin{siglas}
%  \item[ABNT] Associação Brasileira de Normas Técnicas
%  \item[abnTeX] ABsurdas Normas para TeX
%\end{siglas}
% ---

% ---
% inserir lista de símbolos
% ---
%\begin{simbolos}
%  \item[$ \Gamma $] Letra grega Gama
%  \item[$ \Lambda $] Lambda
%  \item[$ \zeta $] Letra grega minúscula zeta
%  \item[$ \in $] Pertence
%\end{simbolos}
% ---

% ---
% inserir o sumario
% ---
\pdfbookmark[0]{\contentsname}{toc}
\tableofcontents*
\cleardoublepage
% ---



% ----------------------------------------------------------
% ELEMENTOS TEXTUAIS
% ----------------------------------------------------------
\textual

% ----------------------------------------------------------
% Introdução (exemplo de capítulo sem numeração, mas presente no Sumário)
% ----------------------------------------------------------
\chapter{Introdução}
%\addcontentsline{toc}{chapter}{Introdução}
% ----------------------------------------------------------


% ----------------------------------------------------------
% PARTE
% ----------------------------------------------------------
%\part{Preparação da pesquisa}
% ----------------------------------------------------------

% ---
% Capitulo com exemplos de comandos inseridos de arquivo externo 
% ---
% ---

% ----------------------------------------------------------
% PARTE
% ----------------------------------------------------------
%\part{Referenciais teóricos}
% ----------------------------------------------------------

% ---
% Capitulo de revisão de literatura
% ---
\chapter{Mapas conceituais}
% ---
%O que são? Quem criou? Quando surgiu? Quais os benefícios para a educação?
% ---
\section{Origem e definição de mapas conceituais}
% ---
O mapas conceituais foram criados na década de 70 durante um estudo de Novak que buscava entender como crianças aprendiam conceitos científicos \cite{Novak2005}. Durante seu estudo, Novak observou uma evolução nas proposições utilizadas pelas crianças, porém teve dificuldade de entender como esta mudança cognitiva ocorreu. Para compreender as mudanças na estruturas cognitivas observadas Novak desenvolveu um método baseado nas três ideias de Ausubel sobre a teoria da aprendizagem \cite{ausubel1963}. A primeira ideia é que novos significados são criados a partir de conceitos e proposições previamente conhecidos pelo aprendiz. A segunda, é que a estrutura cognitiva do indivíduo é organizada hierarquicamente, com conceitos mais gerais ocupando posições superiores, acima dos conceitos mais específicos e complexos. Já a terceira ideia é que durante o processo de aprendizagem os conceitos vão se tornando mais específicos e precisos.

\begin{citacao}[english]
	First, Ausubel sees the development
	of new meanings as building on prior relevant concepts and propositions. Second,
	he sees cognitive structure as organised hierarchically, with more general, more inclusive concepts occupying higher levels in the hierarchy and more specific, less
	inclusive concepts subsumed under the more general concepts. Third, when meaningful learning occurs, relationships between concepts become more explicit, more
	precise, and better integrated with other concepts and propositions\cite{Novak2005}.
\end{citacao}

Assim foi criada essa ferramenta gráfica, que tem como objetivo fazer uma representação do conhecimento. Um mapa conceitual é constituído por interligações entre dois ou mais conceitos através de uma palavra de forma a gerar um significado. Ele é construído ligando-se um conceito a outro, através de um arco que contém uma palavra descritiva, que é usada para criar uma relação significativa entre os conceitos. Além disso, os conceitos mais gerais estão dispostos no topo do mapa e os mais específicos na parte inferior dele. A \autoref{fig_mapconceitual} mostra um exemplo de mapa conceitual.
\begin{figure}[htb]
	\caption{\label{fig_mapconceitual}Um exemplo de mapa conceitual}
	\begin{center}
		\includegraphics[scale=0.3]{mapaconceitual.png}
	\end{center}
	\legend{Fonte: \citeonline[p. 28]{Perin2014}}
\end{figure}
\section{Mapas conceituais e educação}

Os mapas conceituais podem ter diversas aplicações na educação, podem ser utilizados para representar o conhecimento adquirido em uma aula, ou então o conteúdo de um capítulo algum livro lido. Por se tratar de uma representação que é alimentada com novos conceitos conforme eles são adquiridos, pode ser utilizado pelo professor para acompanhar a aprendizagem de um estudante durante todo o período de um curso. Além disso, como os mapas conseguem representar a estrutura cognitiva de um indivíduo, eles podem ser utilizados também como método de avaliação de aprendizagem\cite{Perin2014}.

A confiabilidade da avaliação da aprendizagem tradicional,como textos dissertativos e questões de múltipla escolha, tem sido questionada por pesquisadores e educadores. Nestes tipos de teste apenas os acertos são contabilizados e os erros são descartados, isto pode fazer com que informações importantes para a avaliação sejam desprezadas.
As provas tradicionais não proporcionam a possibilidade do estudante apresentar como construiu o seu aprendizado. Elas conseguem avaliar somente a aprendizagem mecânica, não mostrando como que o aprendiz alterou suas estruturas cognitivas.
Os mapas conceituais tem se destacado como alternativa a avaliação tradicional, pois conseguem demonstrar com facilidade as modificações cognitivas que ocorrem durante o processo de aprendizagem do estudante\cite{Dutra2002}.

\section{Mapas conceituais e tecnologia da informação}

A construção de mapas conceituais é muito simples, uma caneta e um pedaço de papel são ferramentas suficientes para a confecção desse grafo. Porém a tarefa de revisá-lo, armazená-lo, e editá-lo a longo prazo  pode ser muito cansativa e complexa. A introdução do uso de computadores na confecção de mapas pode facilitar a tarefa \cite{Novak2006}.

Os primeiros programas de computadores criados com este objetivo se limitavam a mostrar apenas os mapas na tela, sem oferecer nenhum recurso adicional a ferramenta. Com a popularização do uso de computadores e da Internet houve um aumento na oferta de aplicações que tem como objetivo a construção de mapas gráficos. Porém há uma grande confusão entre usuários e desenvolvedores sobre a diferença entre mapas conceituais, organogramas e mapas mentais, devido a semelhança entre os diagramas\cite{Perin2014}. Um mapa mental, por exemplo, é muito semelhante a um mapa conceitual pois apresenta ligações entre ideias, mas ao contrario de um mapa mental não possui uma palavra descritiva ligando essas ideias e criando organicidade entre as ideias. Portanto uma aplicação de criação de mapas mentais não permite a construção de um mapa conceitual. Segundo a pesquisa de \citeonline{Perin2014}, apenas o software CmapTools pode ser considerada uma ferramenta que oferece suporte a criação de mapas conceituais.

\begin{citacao}[english]
	CmapTools  is  a  software  environment  developed  at  the  Institute for Human and Machine Cognition (IHMC) that empowers users, individually or  collaboratively, to represent their knowledge using concept maps, to share them with peers and colleagues, and to publish them\cite{Canas2004}.
\end{citacao}

Observamos uma carência de softwares que permitam a construção e edição de mapas conceituais. Grande parte das ferramentas para manipulação de mapas gráficos não foram desenvolvidas especificamente para criação de mapas conceituais, e esta foi a grande motivação que tivemos para a realização deste trabalho, visto que, esses mapas consistem em uma ferramenta que tem potencial para inovar o sistema de aprendizagem que conhecemos, principalmente na educação à distância
  

%\lipsum[1]

%\lipsum[2-3]

% ----------------------------------------------------------
% PARTE
% ----------------------------------------------------------
%\part{Resultados}
% ----------------------------------------------------------

% ---
% primeiro capitulo de Resultados
% ---
\chapter{Educação a distância}
% ---
%O que é a educação a distância?
%Qual é a origem da educação a distância?
%Qual é a importancia da educação a distância?
%Quais são as ferramentas de educação a distância?
%O Moodle
%Mapas conceituais e educação a distância.
% ---
\section{Conceituando educação a distância}

A educação a distância é definida de diversas  formas por autores diferentes, apesar disso é possível extrair destas definições distintas um conceito de EAD. Abaixo alguns exemplos dessas definições:
 \begin{itemize}
 	\item Dohemem define EAD como um sistema no qual o aluno estuda sozinho através de um material que lhe é apresentado, sendo o desempenho do estudante acompanhado por um grupo de professores. Esta forma de estudo é possível utilizando meios de comunicação de longa distância.
 	\item Peters conceitua EAD como um método de compartilhar conhecimento através do uso de meios de comunicação, através do qual se torna possível instruir um grande número de alunos.
 	\item segundo Moore EAD é um método no qual os professores agem de forma independente das ações dos estudantes e a comunicação entre ambos deve ser feita através de meios impressos ou eletrônicos.
 	\item o conceito Chaves é de que a EAD é um ensino que ocorre quando há uma separação física entre mestre e aprendiz, sendo utilizadas tecnologias de transmissão de voz, dados e imagens para a comunicação entre ambos.  
 \end{itemize} 
 
 A EAD também é definida no Decreto nº 5.622 de 19 de dezembro de 2005\cite{BRASIL2005}.
 \begin{citacao}
 	Art. 1o  Para os fins deste Decreto, caracteriza-se a educação a distância como modalidade educacional na qual a mediação didático-pedagógica nos processos de ensino e aprendizagem ocorre com a utilização de meios e tecnologias de informação e comunicação, com estudantes e professores desenvolvendo atividades educativas em lugares ou tempos diversos\cite{BRASIL2005}.
 \end{citacao}
 
Em síntese é possível definir a educação a distância como um método de ensino no qual estudante e professor ficam separados fisicamente e a interação e feita através de tecnologias de comunicação (que podem ser dos mais diversos) de maneira a contornar esta separação. Hoje em dia, como sabemos, utiliza-se a Internet como meio de interação entre professor e alunos, mas antes da Internet outras tecnologias de comunicação fizeram parte da historia da educação a distância permitindo que o aprendizado de cursos diversos não precisassem obrigatoriamente de uma sala de aula, o que abriu horizontes para professores e alunos.

\section{Origem da educação a distância}
A educação a distância, ao contrario do que se pensa, surgiu muito antes da existência da Internet, porém a sua origem exata é alvo de controvérsias. Existem compêndios que apontam as epístolas de São Paulo a comunidades cristãs da Ásia Menor como a origem da educação a distância. Estas epístolas foram enviadas no século I e ensinavam as comunidades a viver segundo a doutrina cristã\cite{gouvea2006}.

\citeonline{Sherry1996} aponta os cursos por correspondência europeus como a primeira forma de educação a distância.
Segundo \citeonline{landim1997} inicialmente a educação a distância fazia parte de iniciativas isoladas de alguns educadores, sendo somente institucionalizada na segunda metade do século XIX.

Alguns acontecimentos importante na história da EAD foram a criação da primeira escola de línguas por correspondência em Berlim, no ano de 1856; em 1892 foi criada a Divisão de Ensino por Correspondência na Universidade de Chicago; o nascimento da Tele Escola Primária do Ministério da Cultura e Educação, na Argentina em 1960; entre outros\cite{Alves2011}.

A evolução da EAD pode ser resumida em cinco gerações, de acordo com as tecnologias utilizadas. A primeira geração seria a educação por correspondência. Já a segunda, ocorre na década de 60 e se baseia no uso do rádio e televisão. A terceira, a partir dos anos 80, é representada por softwares multimídia e acesso a Internet como forma de obtenção de conhecimento. A quarta geração consiste no uso de ambientes virtuais de aprendizagem onde a interação entre professores e estudantes é realizada através da Internet. A quinte e última é representada por ambientes de realidade virtual\cite{maia2003}.

Atualmente a educação a distância está relacionada ao uso de um AVA, que nada mais é que um sistema computacional cuja finalidade é prover suporte a atividades mediadas pelas tecnologias de informação e comunicação\cite{Almeida2010}. É um ambiente virtual de aprendizagem que tem como objetivo integrar mídias e recursos, além de apresentar informações de forma organizada e permitir interação entre estudantes, tutores e professores\cite{Franciscato2008}. Um exemplo de um AVA utilizado em algumas instituições de ensino é o Moodle.
   
\section{O Moodle}
O \begin{otherlanguage*}{english}\textit{Modular Object Oriented Distance Learning (Moodle)}\end{otherlanguage*} é um ambiente virtual de aprendizagem que foi criado por Martin Dougiamas em 1999. Ele é uma aplicação \begin{otherlanguage*}{english}\textit{open-source}\end{otherlanguage*}, o que significa que ele é livre para ser instalado, utilizado, modificado e distribuído. Ele é amplamente usado por instituições em todo o mundo e possui uma grande comunidade que contribui para correção de erros e criação de novas ferramentas. \cite{Ribeiro2007}

O \begin{otherlanguage*}{english}\textit{Moodle}\end{otherlanguage*} trabalha com cinco perfis de usuário: administrador, criador de cursos, professor, aluno e visitante. O administrador é o responsável técnico, é ele quem realiza a instalação e configuração do ambiente, além de manter ele funcionando corretamente. Já o criador de cursos tem como responsabilidade a criação e configuração dos cursos disponíveis na plataforma. Por sua vez, o professor tem como função o acompanhamento dos estudantes e a inserção de recursos e tarefas nos cursos. O aluno é quem realiza o curso, é ele quem vai utilizar os recursos e tarefas disponíveis no AVA. Por fim, o visitante é um usuário que só tem acesso as informações disponíveis na tela inicial do sistema. A \autoref{fig_moodle} mostra a interface de um curso no Moodle.


\begin{figure}[htb]
	\caption{\label{fig_moodle}Interface de um curso no Moodle}
	\begin{center}
		\includegraphics[scale=0.2]{moodle.png}
	\end{center}
	\legend{Fonte: Elaborada pelo autor}
\end{figure}
O fato do Moodle ser modular, ou seja, ser composto por módulos instaláveis, configuráveis e estendíveis,  permite que sejam desenvolvidos plugins e componentes que adicionam novas funcionalidades a plataforma, como é o caso do que está sendo desenvolvido neste trabalho.

% ---}


% ---
% segundo capitulo de Resultados
% ---

% ---
%O que é o CMPAAS?
%Quando foi criado?
%Porque foi criado? Qual é a sua importância?
% ---
\chapter{CMPAAS}
\section{Origem}

Vimos como os mapas conceituais são importantes para educação, seja para auxiliar o estudante a aprender novos conceitos, ou como método de avaliação, além de algumas outras aplicações educacionais. Porém, segundo \citeonline{Perin2014} não é só no ensino que há interesse no uso de mapas conceituais e de aplicações para manipulação dos mesmos, tendo o setor empresarial um grande interesse nestas ferramentas. 

Apesar de haver este grande interesse acadêmico e empresarial em ferramentas computacionais para criação e manipulação de mapas, a fragmentação das pesquisas nesta área impede que o desenvolvimento destas aplicações ocorra de forma acelerada. Além disso, a integração de ferramentas já existentes facilitaria o progresso de novas pesquisas.

Se houvesse uma infraestrutura de apoio a criação e manipulação de mapas conceituais o desenvolvimento de novas ferramentas seria facilitado. Um pesquisador poderia criar uma nova solução mais facilmente se não houvesse a necessidade dele se preocupar com toda uma estrutura de gestão de mapas. Por exemplo, alguém que queira desenvolver uma aplicação que avalie mapas conceituais poderia utilizar o conteúdo existente previamente em uma solução computacional de criação e armazenamento de mapas.

Um outro problema apontado por \citeonline{Perin2014} é a dificuldade que a sociedade tem para acessar os resultados das pesquisas.
\begin{citacao}
	Consideramos importante a criação de um mecanismo de
	acesso eficiente aos resultados das pesquisas científicas, para que a comunidade possa contribuir para a evolução delas. O que propomos, portanto, é o lançamento de bases para uma convivência mais estreita entre o mundo acadêmico e a sociedade em geral.\cite{Perin2014}.
\end{citacao}
% ---

Assim foi proposto por \citeonline{Perin2014} a criação de uma plataforma denominada CMPaaS, cujo objetivo é possibilitar que a comunidade em geral acesse o resultado de pesquisas acadêmicas e prover uma infraestrutura que permita que esta comunidade crie e estenda suas funcionalidades.

\section{A plataforma CMaaS}

\subsection{Computação em Nuvem}

O termo computação em nuvem teve sua origem em 2006 durante uma palestra Eric Schmidt sobre como o Google gerenciava seus servidores\cite{taurion2009}. A palavra nuvem é uma abstração para a Internet e toda a sua complexidade de infraestrutura, arquiteturas e componentes. A computação em nuvem é um paradigma no qual o processamento, armazenamento e ferramentas computacionais são oferecidos como um serviço através da Internet. Aplicações baseadas nesta tecnologia possuem a característica de serem extensíveis e facilmente incorporadas a outras que precisem consumir os seus serviços\cite{Perin2014}.

A plataforma CMPaaS usa esta capacidade de expansão e integração da computação em nuvem para oferecer serviços de criação, edição e gestão de mapas conceituais que poderão ser acessados por qualquer usuário no mundo.

\subsection{Arquitetura}

O CMPaaS é uma plataforma orientada a serviços, ou seja, ela utiliza a arquitetura SOA. Isto significa que as funcionalidades oferecidas pela plataforma são facilmente extensíveis e aproveitadas por novas aplicações criadas por qualquer pessoa ao redor do mundo. 

A plataforma foi implementada com a utilização de Web Services, que precisam de uma aplicação com interface para serem consumidos. Assim, o CMPaaS precisa estar associado a um portal que permitirá que o usuário final utilize as ferramentas oferecidas pela plataforma. Este portal foi nomeado como "Portal do Conhecimento"\cite{Perin2014}. 


\begin{figure}[htb]
	\caption{\label{fig_cmpaas}(a) Integração do Portal do Conhecimento com o CMPaaS. (b) Integração do CMPaaS com serviços externos}
	\begin{center}
		\includegraphics[scale=0.3]{cmpaas.png}
	\end{center}
	\legend{Fonte: \citeonline[p. 81]{Perin2014}}
\end{figure}

A \autoref{fig_cmpaas} ilustra a arquitetura do CMPaaS. Podemos observar como as aplicações fornecidas pelo portal utilizam os serviços da plataforma, muitas vezes consumindo mais de um serviço que é oferecido por ela. Além disso, é ilustrado também como um portal externo pode aproveitar os serviços disponíveis no CMPaaS, como é o caso do plugin que será apresentado posteriormente neste trabalho. Ele trata-se de um editor de mapas conceituais que armazena os mapas, criados por ele, na plataforma. Assim ele utiliza dois serviços fornecidos pelo CMPaaS, o de armazenagem de mapas e o de autenticação. 

A estrutura interna da plataforma é composta por cinco camadas\cite{Perin2014}, conforme mostra a \autoref{fig_cmpaas2}.


\begin{figure}[htb]
	\caption{\label{fig_cmpaas2} Arquitetura do CMPaaS}
	\begin{center}
		\includegraphics[scale=0.3]{cmpaas2.png}
	\end{center}
	\legend{Fonte: \citeonline[p. 82]{Perin2014}}
\end{figure}

A camada de serviços externos tem como finalidade prover funcionalidades que serão utilizadas pela comunidade, ou seja, são as aplicações que serão utilizadas por instituições públicas e privadas ou qualquer indivíduo. Os processos de negócio são a camada cuja a responsabilidade é gerenciar todos os processos que ocorrem internamente na plataforma Já a camada de serviços de aplicações internas gerencia os serviços utilizados pelos processos de negócio, esta camada tem como responsabilidade tornar os componentes internos disponíveis para os processos da plataforma. A camada de componentes internos tem como função gerenciar os componentes que produzem todos os serviços da plataforma, é nesta camada que as aplicações de mapas conceituais funcionam. Por último, a camada de serviços de aplicações externas é compostas por serviços externos que são utilizados por componentes internnos\cite{Perin2014}.

\chapter{Tecnologias utilizadas}

\section{JavaScript}

O JavaScript é uma linguagem orientada a objeto criada em 1995 por Brendan Eich que foi idealizada para permitir que pessoas que não são programadoras pudessem estender as funcionalidades de sites de Internet\cite{Richards2010}. Ela é uma linguagem interpratada, e apesar de ter sido criada para desenvolvimento web, ela é uma linguagem de proposito geral, e pode ser usada para o desenvolvimento de qualquer tipo de aplicação\cite{flanagan2006}.

Ela foi escolhida para ser utilizada neste projeto pois o editor de mapas diponível no CMPaaS foi desenvolvido com esta tecnlogia e, além disso, trata-se de uma linguagem compatível com todos os navegadores modernos.  

\section{Json}

O JavaScript Object Notation (Json) é uma linguagem em formato de texto cuja função é a serialização de dados estruturados. A serialização é o processo de tranformar objetos em um fluxo de bytes para ser armazenado em um banco de dados ou disco. Ele pode ser utilizado para representar tipos primitivos, como cadeia de caracteres e numeros, ou tipos estruturados, como objetos e vetores\cite{crockford2006}. Ele é utilizado para intercambio de dados entre os serviços da plataforma CMPaaS.  

\section{PHP}
% ---
PHP é um acrônimo recursivo para PHP: Hypertext Preprocessor, originalmente ele foi criado como uma linguagem de script estruturada cuja finalidade era o desenvolvimento de aplicações com a funcionalidade de geração de HTML dinâmico. Com o passar do tempo a linguagem evoluiu e passou a oferecer recursos para o desenvolvimento orientado a objeto\cite{minetto2007}.

A plataforma Moodle e seus plugins são desenvolvidos em PHP, por isto esta linguagem foi utilizada neste projeto. 

 

\chapter{Plugin}
% ---
%O que é um plugin do moodle?
%Como é feito o plugin?
%Como foi feito o plugin
% ---
\section{O projeto}

O moodle, como vimos anteriormente, é composto por vários componentes, ou módulos, independentes que se relacionam entre si, como qualquer plataforma modular. O escopo deste trabalho é o desenvolvimento de um destes módulos. O plugin que foi criado neste trabalho é do tipo assignment submission plugin, ou seja, é um módulo de submissão de tarefa. O módulo de tarefa permite que um professor crie uma atividade, avaliativa ou não, para os estudantes realizarem.

Este tipo de plugin é divido em três partes:
\begin{itemize}
	\item configurações, que são opções de comportamento do módulo.
	\item sumário de envio, que será visto pelo estudante e avaliadores na tela inicial da tarefa.
	\item segundo Moore EAD é um método no qual os professores agem de forma independente das ações dos estudantes e a comunicação entre ambos deve ser feita através de meios impressos ou eletrônicos.
	\item formulário de envio, que é onde o estudante responde a tarefa.  
\end{itemize} 

O foco deste trabalho é no desenvolvimento do formulário de envio, que é onde entra o editor de mapas conceituais, e no formato no qual o módulo salva o diagrama e o armazena. 

\section{O editor de mapas}
Na criação do formulário de envio, foi utilizado o editor de mapas conceituais disponível no CMPaaS. Ele foi desenvolvido utilizando GoJS, que é uma biblioteca em JavaScript para criação de diagramas para navegadores de Internet.

Este editor foi desenvolvido de forma a salvar os mapas no formato Json, de forma a facilitar que outras aplicações utilizem os dados que ele produz. Ele foi feito desta forma devido a proposta do CMPaaS de permitir a interoperabilidade dos serviços oferecidos.

A primeira parte do trabalho foi realizar um aprimoramento deste editor, adicionando funções para a formatação dos mapas criados por ele. Inicialmente não existia a possibilidade de alterar a fonte da letra dos conceitos ou a cor de fundo dos nós. Uma barra de ferramenta foi adicionada ao editor original para que funções de formatação estivessem disponíveis.

\begin{figure}[htb]
	\caption{\label{fig_barraformacao} Editor com barra de formatação}
	\begin{center}
		\includegraphics[scale=0.5]{barraformacao.png}
	\end{center}
	\legend{Fonte: Elaborada pelo autor}
\end{figure}

% ---
\subsection{Implementação das funções de formatação}
O editor de mapas é dividido em dois componentes, um Model e um Diagram. O Model é o que contém os dados do mapa que está sendo criado, nele os nós e links são descritos por vetores de objetos JavaScript. Já o Diagram é usado para visualizar os dados contidos no modelo.
\begin{figure}[htb]
	\caption{\label{fig_gojs} Componentes do GoJS}
	\begin{center}
		\includegraphics[scale=0.7]{gojs.png}
	\end{center}
	\legend{Fonte: Elaborada pelo autor}
\end{figure}

Cada nó do mapa criado pelo editor é representado por um objeto da classe Node, que por sua vez, é composto por blocos que determinam a sua aparência. Os blocos que o editor utiliza são o Shape e o TextBlock. A classe Shape é utilizada para mostrar uma forma geométrica colorida. Já a classe TextBlock tem como função mostrar um texto. Ambas as classes possuem diversas propriedades que servem para determinar a sua aparência e o seu comportamento no diagrama.

Os trechos de código abaixo são usados para criar um Shape e um TextBlock. A forma geométrica  desenhada é retângulo com largura de 40 pixels, altura de 60 pixels, com margem de 4 pixels e preenchimento na cor vermelha. Já o TextBlock criado é um texto “a Text Block” de cor vermelha.

\begin{figure}[htb]
	\caption{\label{fig_gojselements} Código dos elementos Shape e TextBlock}
	\begin{center}
		\includegraphics[scale=0.6]{gojselements.png}
	\end{center}
	\legend{Fonte: Elaborada pelo autor}
\end{figure}

As novas funcionalidades de formatação adicionadas ao editor modificam as propriedades dos objetos das classes Shape e TextBlock, alterando assim a aparência deles. Foram criados seis botões de formatação com as funções de alterar o tipo de fonte, aumentar o tamanho da fonte, alterar o estilo do texto para negrito, itálico e sublinhado, e para alterar a cor de preenchimento dos nós. Os botões de formatação de fonte alteram a propriedade font da TextBlock, que deve ser uma string  CSS. Já o botão que altera a cor do nó modifica a propriedade fill da Shape.

\section{O plugin para o moodle} 
Para a elaboração deste trabalho foi utilizado um plugin já existente no Moodle como modelo. Todos os plugins de envio de tarefa devem ter uma estrutura de arquivo padrão, esta estrutura será detalhada abaixo.
\begin{itemize}
	\item version.php: este arquivo contém informações sobre a versão do plugin, é utilizado para que o Moodle instale e atualize o plugin corretamente.
	\item settings.php: este arquivo permite que se adicione opções personalizadas para a página configuração do plugin.
	\item lang/en/submission\_nomedoplugin.php:  este é o arquivo de linguagem, ele é usado para internacionalização do plugin.
	\item db/access.php: e utilizada para adicionar capacidades adicionais ao plugin. Este arquivo é opcional, não sendo necessário se o plugin não tiver capacidades adicionais.
	\item db/upgrade.php: este arquivo define a rotina de atualização do plugin. 
	\item db/install.xml: este arquivo define as tabelas de banco de dados que o plugin vai utilizar. 
	\item db/install.php: contem o codigo de instalação do plugin.
	\item db/locallib.php: eh o arquivo mais importante, é ele que define todas as funcionalidades do plugin.    
\end{itemize} 

\subsection{Implementação}
O plugin utilizado como base para o desenvolvimento deste trabalho foi o de envio de texto online, nele o estudante tem a possibilidade responder uma tarefa por meio de um texto escrito diretamente na plataforma. Este projeto foi desenvolvido aplicando engenharia reversa a este plugin, buscando entender a sua funcionalidade e assim descobrir como modifica-lo para tranforma-lo em um editor de mapas.

O módulo de envio de texto online pode ser dividido em duas partes, uma é o editor de textos onde o estudante realiza a sua tarefa e a outra e o sumário da atividade onde tanto o estudante quanto o avaliador tem acesso ao conteúdo da tarefa enviada. O módulo desenvolvido neste trabalho mantém esta estrutura, alterando o editor de textos por um de mapas conceituais e alterando o sumário de forma que o conteúdo apresentado nele passa a ser um json do mapa invés de um texto.

As alterações feitas no plugin de envio de texto online se concentraram no arquivo locallib.php, que é o que determina o comportamento e funcionalidades do módulo. Dentro deste arquivo, as mudanças ocorreram nas funções get\_form\_elements(), que constrói o formulário de envio de tarefa, e no arquivo save(), que realiza a submissão do conteúdo criado pelo estudante.

A primeira parte do trabalho foi substituir o formulário de envio de texto pelo editor de mapas conceituais. Esta modificação foi realizada na função get\_form\_elements(). O código que realizava a inserção do editor de texto foi substituído por um outro, que insere um iframe contendo o editor de mapas do CMPaaS. 

Feito isto, o plugin já apresentava o editor de mapas, porém ainda era necessário salvar o mapa criado no banco de dados do Moodle. Para realizar isto o json do editor contido no iframe precisava ser salvo em algum campo de formulário. Para solucionar isto, foi criado um entrada de dados, oculta na página da tarefa, responsável por receber o json do mapa criado no editor.

Assim, quando o estudante aciona o botão de submissão da tarefa, o json do mapa que ele criou e salvo em um campo de formulário e então armazenado no banco de dados do Moodle.

Para finalizar, foi necessário criar uma forma de o editor de mapas carregar o mapa salvo pelo estudante, para caso ele tivesse interesse em editar um mapa já armazenado. A solução para isto foi realizar o caminho inverso que foi feito ao salvar o mapa criado. Ao carregar o formulário de edição de mapas o json armazenado no banco de dados e carregado em um campo oculto e um código JavaScript se encarrega se obter o conteúdo dele e carregar no mapa.

Então a primeira parte do desenvolvimento do plugin foi concluída, o estudante conseguia criar um mapa no editor, salvar o seu conteúdo na plataforma e editar novamente caso fosse necessário. Porém ainda havia a necessidade de permitir que o avaliador tivesse acesso ao conteúdo gerado pelo aluno e, além disso, proposta do trabalho era a criação de um plugin integrado com o CMPaaS,  portanto a segunda parte do trabalho e realizar esta integração e criar uma forma de o avaliador visualizar o mapa.

\subsection{Integração com o CMPaaS}

\subsection{Instalação e uso}
A pasta contendo os arquivos do plugin deve ser copiada para o diretório do moodle mod/assign/submission/. Após copiar os arquivos para este local é necessário acessar o moodle com um usuário com perfil de administrador. Assim que efetuar login irá aparecer uma mensagem de instalação de plugin, conforme a \autoref{fig_installplugin}.
\begin{figure}[htb]
	\caption{\label{fig_installplugin} Configurando a tarefa para utilizar o plugin}
	\begin{center}
		\includegraphics[scale=0.4]{installplugin.png}
	\end{center}
	\legend{Fonte: Elaborada pelo autor}
\end{figure}

Depois do plugin ser instalado é necessário realizar a configuração da tarefa para utiliza-lo. Para que a tarefa utilize o plugin para submissão de dados é necessário seleciona-lo na tela de configuração da atividade, conforme ilustrado na \autoref{fig_configplugin}.

\begin{figure}[htb]
	\caption{\label{fig_configplugin} Configurando a tarefa para utilizar o plugin}
	\begin{center}
		\includegraphics[scale=0.4]{configplugin.png}
	\end{center}
	\legend{Fonte: Elaborada pelo autor}
\end{figure}

A utilização do plugin é bem simples. Ao acessar uma tarefa o estudante deve clicar no botão de envio de tarefa e ao fazer isto o editor de mapas ira ser mostrado na tela. Após criar ou editar o mapa o aluno deve clicar no botão “Salvar alterações” e o json será salvo no banco de dados da plataforma e aparecerá no sumário, confome \autoref{fig_sumario}.

\begin{figure}[htb]
	\caption{\label{fig_sumario} Configurando a tarefa para utilizar o plugin}
	\begin{center}
		\includegraphics[scale=0.4]{sumario.png}
	\end{center}
	\legend{Fonte: Elaborada pelo autor}
\end{figure}

% ----------------------------------------------------------
% Finaliza a parte no bookmark do PDF
% para que se inicie o bookmark na raiz
% e adiciona espaço de parte no Sumário
% ----------------------------------------------------------
\phantompart

% ---
% Conclusão
% ---
\chapter{Conclusão}
% ---

%\lipsum[31-33]

% ----------------------------------------------------------
% ELEMENTOS PÓS-TEXTUAIS
% ----------------------------------------------------------
\postextual
% ----------------------------------------------------------

% ----------------------------------------------------------
% Referências bibliográficas
% ----------------------------------------------------------
\bibliography{abntex2-modelo-references}

% ----------------------------------------------------------
% Glossário
% ----------------------------------------------------------
%
% Consulte o manual da classe abntex2 para orientações sobre o glossário.
%
%\glossary

% ----------------------------------------------------------
% Apêndices
% ----------------------------------------------------------

% ---
% Inicia os apêndices
% ---
%\begin{apendicesenv}

% Imprime uma página indicando o início dos apêndices
%\partapendices


%\end{apendicesenv}
% ---


% ----------------------------------------------------------
% Anexos
% ----------------------------------------------------------

% ---
% Inicia os anexos
% ---
%\begin{anexosenv}

% Imprime uma página indicando o início dos anexos
%\partanexos


%\end{anexosenv}

%---------------------------------------------------------------------
% INDICE REMISSIVO
%---------------------------------------------------------------------
%\phantompart
%\printindex
%---------------------------------------------------------------------

\end{document}
