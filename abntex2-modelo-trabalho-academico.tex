%% abtex2-modelo-trabalho-academico.tex, v-1.9.6 laurocesar
%% Copyright 2012-2016 by abnTeX2 group at http://www.abntex.net.br/ 
%%
%% This work may be distributed and/or modified under the
%% conditions of the LaTeX Project Public License, either version 1.3
%% of this license or (at your option) any later version.
%% The latest version of this license is in
%%   http://www.latex-project.org/lppl.txt
%% and version 1.3 or later is part of all distributions of LaTeX
%% version 2005/12/01 or later.
%%
%% This work has the LPPL maintenance status `maintained'.
%% 
%% The Current Maintainer of this work is the abnTeX2 team, led
%% by Lauro César Araujo. Further information are available on 
%% http://www.abntex.net.br/
%%
%% This work consists of the files abntex2-modelo-trabalho-academico.tex,
%% abntex2-modelo-include-comandos and abntex2-modelo-references.bib
%%

% ------------------------------------------------------------------------
% ------------------------------------------------------------------------
% abnTeX2: Modelo de Trabalho Academico (tese de doutorado, dissertacao de
% mestrado e trabalhos monograficos em geral) em conformidade com 
% ABNT NBR 14724:2011: Informacao e documentacao - Trabalhos academicos -
% Apresentacao
% ------------------------------------------------------------------------
% ------------------------------------------------------------------------

\documentclass[
	% -- opções da classe memoir --
	12pt,				% tamanho da fonte
	openright,			% capítulos começam em pág ímpar (insere página vazia caso preciso)
	oneside,			% para impressão em recto e verso. Oposto a oneside
	a4paper,			% tamanho do papel. 
	% -- opções da classe abntex2 --
	%chapter=TITLE,		% títulos de capítulos convertidos em letras maiúsculas
	%section=TITLE,		% títulos de seções convertidos em letras maiúsculas
	%subsection=TITLE,	% títulos de subseções convertidos em letras maiúsculas
	%subsubsection=TITLE,% títulos de subsubseções convertidos em letras maiúsculas
	% -- opções do pacote babel --
	english,			% idioma adicional para hifenização
	french,				% idioma adicional para hifenização
	spanish,			% idioma adicional para hifenização
	brazil				% o último idioma é o principal do documento
	]{abntex2}

% ---
% Pacotes básicos 
% ---
\usepackage{lmodern}			% Usa a fonte Latin Modern			
\usepackage[T1]{fontenc}		% Selecao de codigos de fonte.
\usepackage[utf8]{inputenc}		% Codificacao do documento (conversão automática dos acentos)
\usepackage{lastpage}			% Usado pela Ficha catalográfica
\usepackage{indentfirst}		% Indenta o primeiro parágrafo de cada seção.
\usepackage{color}				% Controle das cores
\usepackage{graphicx}			% Inclusão de gráficos
\usepackage{microtype} 			% para melhorias de justificação
% ---
		
% ---
% Pacotes adicionais, usados apenas no âmbito do Modelo Canônico do abnteX2
% ---
\usepackage{lipsum}				% para geração de dummy text
% ---

% ---
% Pacotes de citações
% ---
%\usepackage[brazilian,hyperpageref]{backref}	 % Paginas com as citações na bibl
\usepackage[alf]{abntex2cite}	% Citações padrão ABNT

% --- 
% CONFIGURAÇÕES DE PACOTES
% --- 

% ---
% Configurações do pacote backref
% Usado sem a opção hyperpageref de backref
%\renewcommand{\backrefpagesname}{Citado na(s) página(s):~}
% Texto padrão antes do número das páginas
%\renewcommand{\backref}{}
% Define os textos da citação
%\renewcommand*{\backrefalt}[4]{
%	\ifcase #1 %
%		Nenhuma citação no texto.%
%	\or
%		Citado na página #2.%
%	\else
%		Citado #1 vezes nas páginas #2.%
%	\fi}%
% ---

% ---
% Informações de dados para CAPA e FOLHA DE ROSTO
% ---
\titulo{CMPaaS: Edição  de Mapas  Conceituais em Ambientes Virtuais Moodle}
\autor{Israel Henrique Silva de Lima}
\local{Vitória}
\data{2017}
\orientador{Prof. M.Sc. Wagner de Andrade  Perin}
\coorientador{Prof. Ph.D. Davidson Cury}
\instituicao{
	Universidade Federal do Espírito Santo -- UFES
	\par
	Centro Tecnológico
	\par
	Departamento de Informática}
\tipotrabalho{Monografia (PG)}
% O preambulo deve conter o tipo do trabalho, o objetivo, 
% o nome da instituição e a área de concentração 
\preambulo{Monografia apresentada ao Curso de Engenharia de Computação do Departamento de Informática da Universidade Federal do Espírito Santo, como requisito parcial para obtenção do Grau de Bacharel em Engenharia de Computação.}
% ---


% ---
% Configurações de aparência do PDF final

% alterando o aspecto da cor azul
\definecolor{blue}{RGB}{41,5,195}

% informações do PDF
\makeatletter
\hypersetup{
     	%pagebackref=true,
		pdftitle={\@title}, 
		pdfauthor={\@author},
    	pdfsubject={\imprimirpreambulo},
	    pdfcreator={LaTeX with abnTeX2},
		pdfkeywords={abnt}{latex}{abntex}{abntex2}{trabalho acadêmico}, 
		colorlinks=true,       		% false: boxed links; true: colored links
    	linkcolor=blue,          	% color of internal links
    	citecolor=blue,        		% color of links to bibliography
    	filecolor=magenta,      		% color of file links
		urlcolor=blue,
		bookmarksdepth=4
}
\makeatother
% --- 

% --- 
% Espaçamentos entre linhas e parágrafos 
% --- 

% O tamanho do parágrafo é dado por:
\setlength{\parindent}{1.3cm}

% Controle do espaçamento entre um parágrafo e outro:
\setlength{\parskip}{0.2cm}  % tente também \onelineskip

% ---
% compila o indice
% ---
\makeindex
% ---

% ----
% Início do documento
% ----
\begin{document}

% Seleciona o idioma do documento (conforme pacotes do babel)
%\selectlanguage{english}
\selectlanguage{brazil}

% Retira espaço extra obsoleto entre as frases.
\frenchspacing 

% ----------------------------------------------------------
% ELEMENTOS PRÉ-TEXTUAIS
% ----------------------------------------------------------
% \pretextual

% ---
% Capa
% ---
\imprimircapa
% ---

% ---
% Folha de rosto
% (o * indica que haverá a ficha bibliográfica)
% ---
\imprimirfolhaderosto*
% ---

% ---
% Inserir a ficha bibliografica
% ---

% Isto é um exemplo de Ficha Catalográfica, ou ``Dados internacionais de
% catalogação-na-publicação''. Você pode utilizar este modelo como referência. 
% Porém, provavelmente a biblioteca da sua universidade lhe fornecerá um PDF
% com a ficha catalográfica definitiva após a defesa do trabalho. Quando estiver
% com o documento, salve-o como PDF no diretório do seu projeto e substitua todo
% o conteúdo de implementação deste arquivo pelo comando abaixo:
%
% \begin{fichacatalografica}
%     \includepdf{fig_ficha_catalografica.pdf}
% \end{fichacatalografica}

%\begin{fichacatalografica}
%	\sffamily
%	\vspace*{\fill}					% Posição vertical
%	\begin{center}					% Minipage Centralizado
%	\fbox{\begin{minipage}[c][8cm]{13.5cm}		% Largura
%	\small
%	\imprimirautor
%	%Sobrenome, Nome do autor
%	
%	\hspace{0.5cm} \imprimirtitulo  / \imprimirautor. --
%	\imprimirlocal, \imprimirdata-
%	
%	\hspace{0.5cm} \pageref{LastPage} p. : il. (algumas color.) ; 30 cm.\\
%	
%	\hspace{0.5cm} \imprimirorientadorRotulo~\imprimirorientador\\
%	
%	\hspace{0.5cm}
%	\parbox[t]{\textwidth}{\imprimirtipotrabalho~--~\imprimirinstituicao,
%	\imprimirdata.}\\
%	
%	\hspace{0.5cm}
%		1. Palavra-chave1.
%		2. Palavra-chave2.
%		2. Palavra-chave3.
%		II. Universidade Federal do Espírito Santo.
%		III. Faculdade de xxx.
%		IV. \imprimirtitulo 			
%	\end{minipage}}
%	\end{center}
%\end{fichacatalografica}
% ---



% ---
% Inserir folha de aprovação
% ---

% Isto é um exemplo de Folha de aprovação, elemento obrigatório da NBR
% 14724/2011 (seção 4.2.1.3). Você pode utilizar este modelo até a aprovação
% do trabalho. Após isso, substitua todo o conteúdo deste arquivo por uma
% imagem da página assinada pela banca com o comando abaixo:
%
% \includepdf{folhadeaprovacao_final.pdf}
%
%\begin{folhadeaprovacao}
%
 % \begin{center}
  %  {\ABNTEXchapterfont\large\imprimirautor}
%
 %   \vspace*{\fill}\vspace*{\fill}
  %  \begin{center}
   %   \ABNTEXchapterfont\bfseries\Large\imprimirtitulo
%    \end{center}
%    \vspace*{\fill}
%    
%    \hspace{.45\textwidth}
%    \begin{minipage}{.5\textwidth}
%        \imprimirpreambulo
%    \end{minipage}%
%    \vspace*{\fill}
%   \end{center}
%        
%   Trabalho aprovado. \imprimirlocal, 14 de dezembro de 2016:

 %  \assinatura{\textbf{\imprimirorientador} \\ Orientador} 
  % \assinatura{\textbf{Ph.D. Davidson Cury} \\ Co-orientador}
   %\assinatura{\textbf{M.Sc. Jadir Eduardo Souza Lucas} \\ Examinador}
   %\assinatura{\textbf{Eduardo França} \\ Examinador}
   %\assinatura{\textbf{Professor} \\ Convidado 4}
      
%   \begin{center}
%    \vspace*{0.5cm}
%    {\large\imprimirlocal}
%    \par
%    {\large\imprimirdata}
%    \vspace*{1cm}
%  \end{center}
%  
%\end{folhadeaprovacao}
% ---

% ---
% Dedicatória
% ---
\begin{dedicatoria}
    \vspace*{\fill}
    \centering
    \noindent
    \textit{ Este trabalho é dedicado à todos que me apoiaram durante todo o meu período acadêmico.} \vspace*{\fill}
\end{dedicatoria}
% ---

% ---
% Agradecimentos
% ---
\begin{agradecimentos}
	
Agradeço primeiramente a Deus por todas as graças alcançadas na vida e durante todo o curso.

Aos meus pais, pela educação, apoio emocional, financeiro e tempo dedicado a minha vida. 

Em especial a Jalany, pelo carinho, paciência e apoio durante todo o tempo e principalmente o desenvolvimento desta monografia.

Ao meu orientador Wagner, por toda a ajuda oferecida no elaboração deste trabalho.
 
Ao meu amigo Josmar, pelas longas conversas, orientações e broncas.

Aos meus amigos Piero e Jefferson, pelos momentos de alegria e diversão.

E a todos aqueles que contribuíram direta ou indiretamente nos diversos aspectos de
minha formação acadêmica e pessoal.

\end{agradecimentos}
% ---

% ---
% Epígrafe
% ---
\begin{epigrafe}
     \vspace*{\fill}
 	\begin{flushright}
 		\textit{``Normal people believe that if it ain't broke, don't fix it.\\ Engineers believe that if it ain't broke, it doesn't have enough features yet.\\
 		(Scott Adams)}
 	\end{flushright}
 \end{epigrafe}
% ---

% ---
% RESUMOS
% ---

% resumo em português
\setlength{\absparsep}{18pt} % ajusta o espaçamento dos parágrafos do resumo
\begin{resumo}
Este trabalho  apresenta a concepção e o desenvolvimento de um \textit{plugin} de edição de mapas conceituais para a plataforma Modular Objected-Oriented Dynamic Learning Environment (Moodle) de modo a possibilitar a integração  entre este e a plataforma  de serviços do projeto CMPaaS,  que oferece uma diversidade de serviços para processamento  de mapas conceituais. O objetivo é favorecer ao CMPaaS a integração à uma base de usuários consistente que podem ser potenciais fontes de dados para as ferramentas  e pesquisas desenvolvidas no contexto do projeto CMPaaS. São detalhados a arquitetura, os processos e as tecnologias utilizadas na construção do \textit{plugin}. Uma prova de conceitos é apresentada.
	
 \textbf{Palavras-chave}: Mapas Conceituais. Moodle. CMPaaS.
\end{resumo}

% resumo em inglês
\begin{resumo}[Abstract]
  \begin{otherlanguage*}{english}
    This work presents the design and development of a concept map editing plugin for the Moodle platform in order to aid the integration between this and the services provided by the platform of CMPaaS, a project that offers a variety of services for conceptual mapping. 
    The objective is to aid CMPaaS integration with a consistent user base that may be potential
    sources of data for the tools and research developed in the context of the CMPaaS project.
    The architecture, processes and technologies used in the construction of the plugin are detailed. A proof of concepts is presented.
    
    \vspace{\onelineskip}
 
    \noindent 
    \textbf{Keywords}: Concept Maps. Moodle. CMPaaS.
  \end{otherlanguage*}
\end{resumo}


% ---
% inserir lista de ilustrações
% ---
\pdfbookmark[0]{\listfigurename}{lof}
\listoffigures*
\cleardoublepage
% ---

% ---
% inserir lista de tabelas
% ---
%\pdfbookmark[0]{\listtablename}{lot}
%\listoftables*
%\cleardoublepage
% ---

% ---
% inserir lista de abreviaturas e siglas
% ---
%\begin{siglas}
%  \item[ABNT] Associação Brasileira de Normas Técnicas
%  \item[abnTeX] ABsurdas Normas para TeX
%\end{siglas}
% ---

% ---
% inserir lista de símbolos
% ---
%\begin{simbolos}
%  \item[$ \Gamma $] Letra grega Gama
%  \item[$ \Lambda $] Lambda
%  \item[$ \zeta $] Letra grega minúscula zeta
%  \item[$ \in $] Pertence
%\end{simbolos}
% ---

% ---
% inserir o sumario
% ---
\pdfbookmark[0]{\contentsname}{toc}
\tableofcontents*
\cleardoublepage
% ---



% ----------------------------------------------------------
% ELEMENTOS TEXTUAIS
% ----------------------------------------------------------
\textual

% ----------------------------------------------------------
% Introdução (exemplo de capítulo sem numeração, mas presente no Sumário)
% ----------------------------------------------------------
\chapter{Introdução}
%\addcontentsline{toc}{chapter}{Introdução}
% ----------------------------------------------------------

Mapas conceituais são uma ferramenta gráfica muito utilizada na representação de conhecimento acerca de um dado domínio de problema. São fáceis de utilizar e possuem ampla aplicação na educação, podendo ser usados para apresentar o conteúdo de uma aula, de um livro, capítulo ou utilizado como método de avaliação.

Por sua simplicidade no processo de elaboração, mapas podem ser elaborados com lápis e papel, porém é possível alcançar novas aplicações quando da utilização de recursos computacionais, tais como a utilização de recursos dotados de inteligência artificial para análise de mapas \cite{de2013construindo} e mineração de dados \cite{yoo2012mining}.

Pensando nisso, pesquisadores da Universidade Federal do Espírito Santo especificaram e desenvolveram uma plataforma denominada CMPaaS. Esta plataforma possui a finalidade de fornecer um ambiente favorável à disposição e integração de coleções de ferramentas voltadas para a gestão e uso inteligente de mapas conceituais que são frutos de pesquisas desenvolvidas por diversos estudantes deste grupo, além de pessoas externas ao projeto.

O CMPaaS concentra diversas ferramentas de mapas conceituais desenvolvidas em projetos do laboratório de educação da UFES, e muitas delas necessitam levantar dados para comprovação de hipóteses. Um exemplo de ferramenta é o iMap, que foi desenvolvido por \citeonline{Perin2014} em sua dissertação de mestrado.  

O iMap é uma ferramenta que utiliza inteligência computacional para fornecer uma camada de abstração entre um mapa conceitual e um avaliador. Assim é possível a um avaliador examinar um mapa conceitual através de perguntas em linguagem natural, sem haver a necessidade de um exame visual dos conceitos e relacionamentos presentes no mapa. No entanto a real capacidade desta ferramenta não pode ser efetivamente comprovada devido a escassez de utilizadores e de dados de entrada.

Nesta linha, este trabalho acredita que uma forma efetiva de ampliar as fontes de dados para a plataforma do CMPaaS é favorecer a integração com sistemas que já possuam base de usuários consolidadas. Assim, propõe-se a criação de uma ferramenta que permita o uso dos mapas nas plataformas utilizadas no ensino a distância, já que estes são nichos grandes e, a julgar pelos avanços na utilização de recursos computacionais na educação, tendem a se ampliar nos próximos anos \cite{ribeiro2012}.

Assim, este projeto visa o desenvolvimento de um plugin para construção de mapas conceituais dentro do gerenciador de cursos Moodle que permita o compartilhamento dos mapas produzidos com a plataforma CMPaaS.

Para alcançar esses objetivos, foi realizada uma revisão bibliográfica aprofundada sobre mapas conceituais e a sua importância para a educação; também um  estudo do funcionamento e arquitetura do Moodle para entender como podem ser desenvolvidos \textit{plugins} para adicionar novos recursos à plataforma;  foi feito um estudo sobre soluções de integração de sistemas para entender como seria possível a realização da interoperabilidade entre o Moodle e o CMPaaS. Por fim, o trabalho apresenta uma arquitetura conceitual e uma prova de conceitos para explorar as funcionalidades do plugin após a sua implementação.

\section{Motivação} 
 
O CMPaaS é uma plataforma baseada em serviços web que agrega diversas ferramentas de manipulação de mapas conceituais. Ferramentas estas que carecem de uma base de usuários consolidada para geração de conteúdo.

A motivação para esse projeto de graduação é portanto, buscar uma forma de trazer para o CMPaaS uma base de usuários que possa alimenta-lo com conteúdo a ser consumido pelas ferramentas que integram a plataforma.

Sendo assim este trabalho propõe a adoção de uma solução de integração de sistemas que leve a interoperabilidade entre o CMPaaS e uma plataforma que possua uma grande base de usuários.

\section{Metodologia de trabalho}

Com o objetivo geral deste trabalho definido, para alcança-lo foram estipulados os seguintes passos:

\begin{itemize}
	\item Estudar o que são mapas conceituais e suas aplicações na educação.
	\item Estudar a arquitetura do Moodle e seus \textit{plugins}.
	\item Estudar técnicas e arquiteturas de integração de sistemas.
	\item Analisar a arquitetura da plataforma CMPaaS a fim de entender como realizar a integração dela com outras plataformas.
	\item Desenvolver um novo \textit{plugin} para o Moodle segundo a arquitetura da plataforma.   
\end{itemize} 
 
\section{Resultados Esperados}

O resultado esperado deste trabalho é demonstrar a possibilidade de integração dos serviços providos pelo CMPaaS com outras plataformas. E com isto aumentar o uso das ferramentas presentes na plataforma e o conteúdo existente na mesma. 

% ----------------------------------------------------------
% PARTE
% ----------------------------------------------------------
%\part{Preparação da pesquisa}
% ----------------------------------------------------------

% ---
% Capitulo com exemplos de comandos inseridos de arquivo externo 
% ---
% ---

% ----------------------------------------------------------
% PARTE
% ----------------------------------------------------------
%\part{Referenciais teóricos}
% ----------------------------------------------------------

% ---
% Capitulo de revisão de literatura
% ---
\chapter{Mapas conceituais e educação}\label{cap-maps}
% ---
%O que são? Quem criou? Quando surgiu? Quais os benefícios para a educação?
% ---

Este capítulo apresenta a origem e a definição de mapas conceituais e também a sua importância para a educação. E como a tecnologia da informação pode contribuir para melhorar a usabilidade dos mapas conceituais. e será apresentado o CMPaaS, uma plataforma online que oferece serviços para mapas conceituais. Fala também sobre a plataforma CMPaaS, que oferece serviços para mapas conceituais, e os benefícios de sua integração com o Moodle.


\section{Origem e definição de mapas conceituais}
% ---
Os mapas conceituais constituem uma ferramenta para a representação do conhecimento. Foram criados na década de 70 durante um estudo de Novak que buscava entender como crianças aprendiam conceitos científicos \cite{Novak2005}. Durante seu estudo, Novak observou uma evolução nas proposições utilizadas pelas crianças, porém teve dificuldade de entender como esta mudança cognitiva ocorreu. Para compreender tal mudança, Novak desenvolveu um método baseado nas três ideias de Ausubel sobre a teoria da aprendizagem \cite{ausubel1963}. A primeira ideia é que novos significados são criados a partir de conceitos e proposições previamente conhecidos pelo aprendiz. A segunda, é que a estrutura cognitiva do indivíduo é organizada hierarquicamente, com conceitos mais gerais ocupando posições superiores, acima dos conceitos mais específicos. Já a terceira ideia é que durante o processo de aprendizagem os conceitos vão se tornando mais específicos e precisos.

Um mapa conceitual é constituído por interligações entre dois ou mais conceitos através de uma palavra de forma a gerar um significado. Ele é construído ligando-se um conceito a outro, através de um arco que contém uma palavra descritiva, que é usada para criar uma relação significativa entre os conceitos. Além disso, os conceitos mais gerais estão dispostos no topo do mapa e os mais específicos na parte inferior dele. A \autoref{fig_mapconceitual} mostra um exemplo de mapa conceitual.
\begin{figure}[htb]
	\caption{\label{fig_mapconceitual}Um exemplo de mapa conceitual}
	\begin{center}
		\includegraphics[scale=0.3]{mapaconceitual.png}
	\end{center}
	\legend{Fonte: \citeonline[p. 28]{Perin2014}}
\end{figure}
\section{Mapas conceituais e educação}\label{sec-mapeduc}

Os mapas conceituais podem ter diversas aplicações na educação. Podem por exemplo ser utilizados para representar o conhecimento adquirido em uma aula, ou então o conteúdo de um capítulo de um livro. Por se tratar de uma representação que é alimentada com novos conceitos conforme eles são adquiridos, pode ser utilizado pelo professor para acompanhar a aprendizagem de um estudante durante todo o período de um curso. Ademais, como os mapas conseguem representar a estrutura cognitiva de um indivíduo, eles podem ser utilizados também como método de avaliação de aprendizagem \cite{Perin2014}.

A confiabilidade da avaliação da aprendizagem tradicional, como textos dissertativos e questões de múltipla escolha, tem sido questionada por pesquisadores e educadores. Nestes tipos de teste apenas os acertos são contabilizados e os erros são descartados, implicando em importantes informações para a avaliação da aprendizagem serem desprezadas.
As provas tradicionais não proporcionam a possibilidade do estudante apresentar como construiu o seu aprendizado. Elas conseguem avaliar somente a aprendizagem mecânica, não mostrando como que o aprendiz alterou suas estruturas cognitivas.
Os mapas conceituais tem se destacado como alternativa a avaliação tradicional, pois conseguem demonstrar com facilidade as modificações cognitivas que ocorrem durante o processo de aprendizagem do estudante \cite{Dutra2002}.

Neste contexto, o \textit{plugin} que será desenvolvido neste trabalho tem como finalidade permitir que estudantes de cursos gerenciados pelo Moodle respondam tarefas avaliativas com a elaboração de mapas conceituais. Assim poderá ampliar muito o uso de mapas como método de avaliação, já que o Moodle é uma plataforma utilizada por diversas instituições.

\section{Mapas conceituais e tecnologia da informação}

A construção de mapas conceituais é muito simples, uma caneta e um pedaço de papel são ferramentas suficientes para a confecção desse grafo. Porém a tarefa de revisá-lo, armazená-lo, e editá-lo a longo prazo  pode ser muito cansativa e complexa. A introdução do uso de computadores na confecção de mapas pode facilitar a tarefa \cite{Novak2006}.

Os primeiros programas de computadores criados com este objetivo se limitavam a mostrar apenas os mapas na tela, sem oferecer nenhum recurso adicional a ferramenta. Com a popularização do uso de computadores e da Internet houve um aumento na oferta de aplicações que tem como objetivo a construção de mapas gráficos. Porém há uma grande confusão entre usuários e desenvolvedores sobre a diferença entre mapas conceituais, organogramas e mapas mentais, devido a semelhança entre os diagramas \cite{Perin2014}. Um mapa mental, por exemplo, é muito semelhante a um mapa conceitual pois apresenta ligações entre ideias, mas ao contrario de um mapa mental não possui uma palavra descritiva ligando essas ideias e criando organicidade entre as ideias. Portanto uma aplicação de criação de mapas mentais não permite a construção de um mapa conceitual. 

Na seção 2.3 de sua dissertação de mestrado, \citeonline{Perin2014} realizou uma pesquisa sobre o estado da prática em mapas conceituais. Nela foi feito um levantamento sobre softwares que permitem a criação de mapas conceituais e quais são as funcionalidades oferecidas por eles. Nesta pesquisa foram avaliadas 16 ferramentas de edição de mapas, das quais apenas o CMapTools foi desenvolvido com o objetivo de oferecer suporte específico para mapas conceituais.

Percebendo esta carência, um grupo de pesquisadores do Laboratório de Informática na Educação da Universidade Federal do Espírito Santo (LIED-DI-UFES) propôs uma plataforma de serviços para mapas conceituais, conhecida por CMPaaS \cite{Perin2016} cujo objetivo é fornecer serviços avançados para facilitar os processos e as aplicações dos mapas conceituais, tanto no ensino quanto no mercado. Neste trabalho, interessa-nos a criação de uma ferramenta integrada ao CMPaaS que proporcione a aplicação dos mapas conceituais na Educação a Distância. 

\subsection{CMPaaS}

Na \autoref{sec-mapeduc} vimos como os mapas conceituais são importantes para educação, seja para auxiliar o estudante a aprender novos conceitos, ou como método de avaliação, além de algumas outras aplicações educacionais. 

Apesar de haver este grande interesse acadêmico em ferramentas computacionais para criação e manipulação de mapas, a fragmentação das pesquisas nesta área impede que o desenvolvimento destas aplicações ocorra de forma acelerada. Assim, a integração de ferramentas já existentes facilitaria o progresso de novas pesquisas.

Se houvesse uma infraestrutura de apoio a criação e manipulação de mapas conceituais o desenvolvimento de novas ferramentas seria facilitado. Um pesquisador poderia criar uma nova solução mais facilmente se não houvesse a necessidade dele se preocupar com toda uma estrutura de gestão de mapas. Por exemplo, alguém que queira desenvolver uma aplicação que avalie mapas conceituais poderia utilizar o conteúdo existente previamente em uma solução computacional de criação e armazenamento de mapas.

Um outro problema apontado por \citeonline{Perin2014} é a dificuldade que a sociedade tem para acessar os resultados das pesquisas.
\begin{citacao}
	Consideramos importante a criação de um mecanismo de
	acesso eficiente aos resultados das pesquisas científicas, para que a comunidade possa contribuir para a evolução delas. O que propomos, portanto, é o lançamento de bases para uma convivência mais estreita entre o mundo acadêmico e a sociedade em geral \cite{Perin2014}.
\end{citacao}
% ---

Assim foi proposto por \citeonline{Perin2014} a criação de uma plataforma denominada CMPaaS, cujo objetivo é possibilitar que a comunidade em geral acesse o resultado de pesquisas acadêmicas e prover uma infraestrutura que permita que esta comunidade crie e estenda suas funcionalidades.

Os serviços do CMPaaS são oferecidos através da Internet, podendo ser acessado de qualquer plataforma. Mas é necessário que exista um portal ou site associado a plataforma CMPaaS para que as ferramentas possam ser acessadas através de um navegador de Internet. Este portal foi nomeado como "Portal do Conhecimento" \cite{Perin2014}. 


\begin{figure}[htb]
	\caption{\label{fig_cmpaas}(a) Integração do Portal do Conhecimento com o CMPaaS. (b) Integração do CMPaaS com serviços externos}
	\begin{center}
		\includegraphics[scale=0.3]{cmpaas.png}
	\end{center}
	\legend{Fonte: \citeonline[p. 81]{Perin2014}}
\end{figure}

A \autoref{fig_cmpaas} ilustra a arquitetura do CMPaaS e como as aplicações fornecidas pelo portal utilizam os serviços da plataforma, muitas vezes consumindo mais de um dos serviços oferecidos. Além disso, é ilustrado também como um portal externo pode aproveitar os serviços disponíveis no CMPaaS, como é o caso do plugin que será apresentado posteriormente neste trabalho. Trata-se de um editor de mapas conceituais que armazena os mapas na plataforma. Assim são utilizados dois serviços fornecidos pelo CMPaaS: o de armazenagem de mapas e o de autenticação.

Atualmente a plataforma CMPaaS oferece os seguintes serviços:

\begin{itemize}
	\item Criação, edição e formatação de mapas conceituais.
	\item Serviço de cadastramento e autenticação de usuários.
	\item Persistência e repositório de mapas.
	\item Controle de versão de mapas conceituais.
	\item Importação e exportação de mapas para o CmapTools
	\item Serviço de inferência para mapas, que permite que o usuário elabore perguntas sobre os mapas em linguagem natural.
	\item Criação de ontologias rasas a partir de um mapa.
	\item Geração automática de mapas a partir de textos.
	\item Serviço de mesclagem de mapas conceituais.
	\item Validação estrutural de mapas.
\end{itemize}

 Como um dos objetivos do CMPaaS é oferecer um banco de dados de mapas conceituais para ser utilizado como base para pesquisas e criação de novas ferramentas, é desejável que seu serviço de repositório de mapas seja constantemente utilizado e incrementado com novos conteúdos. Pensando nisto este trabalho visa realizar a integração do Moodle com o CMPaaS, abrindo a possibilidade deste utilizar a ampla base de usuários do Moodle para alimentar o seu repositório de mapas. 

%\lipsum[1]

%\lipsum[2-3]

% ----------------------------------------------------------
% PARTE
% ----------------------------------------------------------
%\part{Resultados}
% ----------------------------------------------------------

% ---
% primeiro capitulo de Resultados
% ---
%\chapter{Educação a distância}
% ---
%O que é a educação a distância?
%Qual é a origem da educação a distância?
%Qual é a importancia da educação a distância?
%Quais são as ferramentas de educação a distância?
%O Moodle
%Mapas conceituais e educação a distância.
% ---
\section{Moodle e Educação a Distância (EaD)}\label{cap-ead}

Segundo \citeonline{chaves1999} EaD é um ensino que ocorre quando há uma separação física entre mestre e aprendiz, sendo utilizadas tecnologias de transmissão de voz, dados e imagens para a comunicação entre ambos. A EaD também é definida no Decreto nº 5.622 de 19 de dezembro de 2005 \cite{BRASIL2005}.
 \begin{citacao}
 	Art. 1o  Para os fins deste Decreto, caracteriza-se a educação a distância como modalidade educacional na qual a mediação didático-pedagógica nos processos de ensino e aprendizagem ocorre com a utilização de meios e tecnologias de informação e comunicação, com estudantes e professores desenvolvendo atividades educativas em lugares ou tempos diversos \cite{BRASIL2005}.
 \end{citacao}
 
Em síntese é possível definir a educação a distância como um método de ensino no qual estudante e professor ficam separados fisicamente e a interação e feita através de tecnologias de comunicação (que podem ser dos mais diversos) de maneira a contornar esta separação.

Atualmente a educação a distância está relacionada ao uso de um Ambiente Virtual de Aprendizagem (AVA), que nada mais é que um sistema computacional cuja finalidade é prover suporte a atividades mediadas pelas tecnologias de informação e comunicação \cite{Almeida2010}. É um ambiente que tem como objetivo integrar mídias e recursos, além de apresentar informações de forma organizada e permitir interação entre estudantes, tutores e professores \cite{Franciscato2008}. Um exemplo de um AVA são os cursos que podem ser criados e gerenciados no Moodle.
   
O Moodle é um sistema gerenciador de cursos que foi criado por Martin Dougiamas em 1999. Ele é uma aplicação \begin{otherlanguage*}{english}\textit{open-source}\end{otherlanguage*}, o que significa que ele é livre para ser instalado, utilizado, modificado e distribuído \cite{Dougiamas2003}. 

O Moodle trabalha por padrão com cinco perfis de usuário: administrador, criador de cursos, professor, aluno e visitante. O administrador é o responsável técnico, é ele quem realiza a instalação e configuração do ambiente, além de manter ele funcionando corretamente. Já o criador de cursos tem como responsabilidade a criação e configuração dos cursos disponíveis na plataforma. Por sua vez, o professor tem como função o acompanhamento dos estudantes e a inserção de recursos e tarefas nos cursos. O aluno é quem realiza o curso, é ele quem vai utilizar os recursos e tarefas disponíveis no AVA. Por fim, o visitante é um usuário que só tem acesso as informações disponíveis na tela inicial do sistema. A \autoref{fig_moodle} mostra a interface de um curso no Moodle.


\begin{figure}[htb]
	\caption{\label{fig_moodle}Interface de um curso no Moodle}
	\begin{center}
		\includegraphics[scale=0.2]{moodle.png}
	\end{center}
	\legend{Fonte: Elaborada pelo autor}
\end{figure}


Ele foi escolhido para ser utilizado neste trabalho pois é amplamente usado por instituições em todo o mundo, possuindo mais de 73 mil sites registrados em 232 países \cite{MoodleStat2016}, e possui uma grande comunidade que contribui para correção de erros e criação de novas ferramentas. Além disso ele é utilizado para gerenciar os AVAs da Ufes, e é muito utilizado também em diversas instituições no país, sendo o Brasil o terceiro maior utilizador da plataforma, conforme pode ser visto na \autoref{tabela-moodle}.



O Moodle é composto por módulos instaláveis, configuráveis e estendíveis. Assim, é possível desenvolver \textit{plugins} e componentes que adicionam novas funcionalidades a plataforma, como é o caso do que está sendo desenvolvido neste trabalho.

O que propomos neste projeto é, portanto, a criação de um novo \textit{plugin} para o Moodle que permita ao usuário construir mapas dentro desta plataforma e utilizar o serviço de repositório de mapas oferecido pelo CMPaaS. Para isto será necessário conhecer as tecnologias que permitem a integração de sistemas modernos. Este tópico será abordado com maiores detalhes no próximo capítulo.  

\begin{table}[htb]
	\IBGEtab{%
		\caption{Os 10 países que mais utilizam o Moodle.}%
		\label{tabela-moodle}
	}{%
	\begin{tabular}{ccc}
		\toprule
		País & Sites Registrados \\
		\midrule \midrule
		Estados Unidos & 10.131 \\
		\midrule 
		Espanha & 7.067 \\
		\midrule 
		Brasil & 4.401 \\
		\midrule 
		Reino Unido & 3.486 \\
		\midrule 
		México & 3.464 \\
		\midrule 
		Alemanha & 2.444 \\
		\midrule 
		Itália & 2.414 \\
		\midrule 
		Austrália & 2.324 \\
		\midrule 
		Colômbia & 2.264 \\
		\midrule 
		Rússia & 1.993 \\
		\bottomrule
	\end{tabular}%
}{%
\fonte{\citeonline{MoodleStat2016}}%
}
\end{table}

% ---}


% ---
% segundo capitulo de Resultados
% ---

\chapter{Tecnologias para Integração de Sistemas}

Nos últimos anos muitas organizações têm utilizado aplicações para dar suporte aos seus negócios e por esse motivo sistemas de informação têm se tornado um de seus pilares de funcionamento \cite{martins2006integraccao}. O aumento do uso da tecnologia de informação juntamente com a busca pelo controle e flexibilização de informações fez crescer a demanda por soluções que viabilizassem a partilha de informações e de funcionalidades de aplicações \cite{edwards2000application}. Aplicações estas, que em grande parte dos casos, foram desenvolvidas de forma customizada e sem a capacidade de integração com outros sistemas. Neste contexto, as organizações seguiram diversas abordagens para a integração dos seus sistemas de informação, criando então soluções sem nenhuma norma técnica, devido a inexistência das mesmas. 

O crescimento da necessidade de soluções de integração de sistemas levou ao desenvolvimento de normas técnicas e protocolos direcionados a interoperabilidade de sistemas, surgindo então um cenário desafiador onde as instituições precisam escolher entre diversas normas e técnicas de integração de sistemas, muitas delas incompatíveis entre si \cite{martins2006integraccao}. 

Dentre estas tecnologias se destacam Simple Object Access Protocol (SOAP) e Representational State Transfer (REST), que obtiveram ampla aceitação no mercado pois conseguem se aproveitar do protocolo Hypertext Transfer Protocol (HTTP). Com o REST tendo destaque por ser um modelo de arquitetura simples e robusto, que é capaz de atenter requisitos de integração complexos e com pouco acoplamento \cite{Lusa}.   

\section{HTTP}

O HTTP é o protocolo base de comunicação de aplicações Web e vem sendo utilizado na World-Wide Web desde 1990. Na sua versão inicial ele se tratava de um protocolo simples, utilizado para transmitir apenas dados brutos, mas foi aprimorado de forma a permitir a transmissão de mensagens contendo metadados sobre a informação transmitida, e também modificadores para semânticas de requisições e respostas. Assim, hoje ele é o protocolo padrão dos navegadores Web, sendo utilizado por diversas aplicações na Internet \cite{fielding1999hypertext} . 

No HTTP a comunicação é realizada através de requisições e respostas entre clientes e servidores. Um cliente, que pode ser um navegador web, requisita um recurso a um servidor enviando uma mensagem contendo um cabeçalho. Esta mensagem deve ser enviada para uma URI e o servidor então responde com um recurso ou com um outro cabeçalho. 

A \autoref{fig_http_header} é uma captura de tela que mostra o cabeçalho de uma requisição http. Nela é possível observar a URI, no caso /api/users/, e o metódo da requisição, que no exemplo é GET. Também é possível observar o campo authorization, que é uma chave de autorização usada na autenticação do usuário que realizou a requisição. O servidor responde a esta requisição como um outro cabeçalho e também um conteúdo, que no caso é uma lista de usuários. 

\begin{figure}[htb]
	\caption{\label{fig_http_header} Cabeçalho de uma requisição HTTP}
	\begin{center}
		\includegraphics[scale=0.7]{http_header.png}
	\end{center}
	\legend{Fonte: Elaborado pelo autor}
\end{figure}

\begin{figure}[htb]
	\caption{\label{fig_http_response} Cabeçalho de uma resposta HTTP}
	\begin{center}
		\includegraphics[scale=0.7]{http_response.png}
	\end{center}
	\legend{Fonte: Elaborado pelo autor}
\end{figure}

Um cabeçalho de resposta HTTP pode ser visto na \autoref{fig_http_response}. Este cabeçalho possui um código que serve para identificar se a requisição foi concluida com sucesso. No cabeçalho deste exemplo, o código de resposta 200 indica que a requisição foi concluida com sucesso. Já um código 401 indicaria que URI acessada exige autenticação do usuário. O campo \textit{date} informa a data e hora em que a mensagem foi enviada. Já o campo \textit{server} contém o nome do servidor que respondeu a requisição. O cabeçalho informa em \textit{Content-type} o tipo de conteúdo da mensamgeme em \textit{Allow} os métodos HTTP que são aceitos.

O HTTP é um protocolo stateless, ou seja, sem estado. Isto significa que ele não guarda informações entre requisiçoes diferentes. Caso seja necessário armazenar dados é preciso utilizar alguma outra tecnologia em conjunto com o HTTP, como cookies ou variáveis de sessão.

O protocolo possui métodos para identificar o tipo de requisição que está sendo realizada. Estes métodos são GET, POST, DELETE, PUT, HEAD, OPTIONS, TRACE e CONNECT. 

Neste trabalho os métodos utilizados são os que explicarei abaixo:

\begin{itemize}
	\item GET: utilizado para requisitar informações sobre um recurso.
	\item POST: tem como funcionalidade a criação de um novo recurso ou o processamento de informações. 
	\item PUT: é utilizado para atualizar um recurso já existente ou para criar um novo.
	\item DELETE: apaga um recurso.     
\end{itemize} 

%\section{Soap}
%
%SOAP é um protocolo para troca de mensagens estruturadas em sistemas distribuídos. Um sistema distribuído consiste em um conjunto de computadores independentes, dispostos em uma rede, que se comunicam e coordenam suas ações através de mensagens. SOAP é baseado em XML e precisa de um protocolo da camada de aplicação de rede para realizar a comunicação, como o SMTP ou HTTP.  Devido ao HTTP ser um protocolo amplamente utilizado na Web ele é o que teve maior aceitação para ser utilizado com SOAP. Uma mensagem deste protocolo consiste em um envelope que é anexado a um cabeçalho HTTP. Este envelope é composto por dois componentes, o Header e o Body, como mostra figura x que mostra  a estrutura de uma mensagem SOAP enviada com a utilização do protocolo HTTP. 
%O envelope é um elemento obrigatório da mensagem, ele pode conter declarações de namespace e também atributos adicionais que definem como os dados serão representados no documento XML. O header compoe o envelope e é opcional e, quando utilizado deve ser o primeiro elemento do envelope. Ja o Body é um elemento obrigatório e contém a informação que deve ser transmitida, que também é conhecida com payload. 

\section{REST}

% O que é?
% Relação com o HTTP
% Recursos e caracteristicas

REST é um estilo de arquitetura que representa a abstração da interações de aplicações web. É um conjunto de restrições de arquitetura que busca minimizar o tempo de resposta na comunicação entre aplicações e ao mesmo tempo aumentar a independência na implementação de componentes. O REST foi criado por Roy T. \citeonline{fielding2000rest} em sua tese de doutorado e faz uso de tecnologias amplamente utilizadas, como o protocolo HTTP, para orientar a criação de aplicações baseadas nos fundamentos da Web. A arquitetura REST descreve interfaces que utilizam o HTTP para transmissão de dados sem a utilização de mensagens adicionais como ocorre com SOAP \cite{dal2009web}.

Apesar de não ser obrigatório o uso do protocolo HTTP, ele é o único protocolo conhecido que é  totalmente compatível com a tecnologia. Sendo o REST orientado às boas práticas de uso do HTTP, como o  uso correto de seus métodos, a criação de URL’s e códigos de resposta padronizados e o uso adequado de cabeçalhos.  Como o HTTP é um dos protocolos mais utilizados na web a adoção do REST pela comunidade de desenvolvedores foi grande e rápida.	

Ao se navegar na web todo documento acessível é chamado de recurso e isto não é diferente no REST. Este estilo de arquitetura é orientado a recursos, que são os conjuntos de dados que estarão disponível em uma aplicação web que implementa o REST. Cada recurso necessita de uma identificação única ou URI, que irá nomea-lo e fornecer um caminho para acessa-lo.

A interação com as URIs é realizada através dos métodos HTTP. Esta interação é guiada por uma regra na qual as URIs são substantivos e os métodos são verbos. E portanto os metodos sao aqueles que realizam alteraçoõs nos recusrsos identificados pelas URIs. O método GET, por exemplo, é utilizado para se obter dados de um recurso, já o método DELETE é tem como funcionalidade apagar um recurso. 

O REST também padroniza também o uso dos códigos de respostas do HTTP. Estes códigos são divididos em cinco grupos, e cada grupo possui um significado. Códigos 1xx são informacionais, códigos 2xx indicam sucesso, codigos 3xx informam um redirecionamento, 4xx indicam erros do cliente e 5xx erros de servidor.

\subsection{Princípios da arquitetura REST}

Em sua tese de doutorado \citeonline{fielding2000rest} estabeleceu príncipios que devem ser repeitados na implementação da arquitetura REST. E com isto as seguintes características devem ser respeitadas:

\begin{itemize}
	\item \textbf{Arquitetura cliente-servidor:} é um caracterísica comum em serviços Web. O cliente inicia a comunicação requisitando um serviço ou recurso a um servidor. O servidor escuta e processa a requisição feita pelo cliente e então o envia uma resposta. Esta resposta pode ser o serviço ou recurso que o cliente requisitou, ou então pode ser uma rejeição à requisição realizada pelo cliente.  
	\item \textbf{Stateless:} a comunicação entre cliente e o servidor é sem estado. Isto significa que a interação entre ambos é feita sem que seja armazenado qualquer tipo de estado no servidor. Ou seja, cada requisição é independente uma  da outra e deve conter toda informação necessária para a comunicação.  
	\item \textbf{\textit{Cacheable}:} as respostas enviadas pelo servidor podem ser armazenadas em um cache no cliente. Assim, quando uma requisição semelhante é realizada o dado é acessado no cache do cliente, não sendo necessário receber uma nova resposta do servidor. 
	\item \textbf{Interface Uniforme:} a interação entre o cliente e o servidor é realizada com a utilização de uma interface padronizada. Esta inferface deve identificar os recursos e manipula-los através de representações. Neste sentido o HTTP é o protocolo mais indicado para ser utilizado com o REST, pois possui uma interface uniforme com métodos bem definidos para realizar operações básicas.
	\item \textbf{Multicamada:} a arquitetura REST preve a divisão das funcionalidades de um sistema em camadas. Cada uma delas tendo a possibilidade de se comunicar com a camada adjacente.
	\item \textbf{\textit{Code On Demand}:}  esta característica que permite obter um código do servidor que será executa diretamente no lado do cliente. Um exemplo de uso é um código JavaScript que é baixado do servidor e executado por um navegador Web no cliente.           
\end{itemize}  

% ---
%O que é o CMPAAS?
%Quando foi criado?
%Porque foi criado? Qual é a sua importância?
% ---


 

\chapter{Projeto}\label{cap-projeto}
% ---
%O que é um plugin do moodle?
%Como é feito o plugin?
%Como foi feito o plugin
% ---

%Josmar: só estar desatualizado é motivo suficiente para escrever um plugin do zero? Por que não apenas atualizar o plugin desatualizado? Faltou descrever aqui (talvez esse plugin desatualizado não é open source, ou então só mesmo atualizando ainda faltaria a integração com CMPaaS). Elabore melhor essa critica, pois ela aparentemente está tentando embasar a criação de um novo plugin (o seu).

Neste capítulo será apresentada a arquitetura do \textit{plugin} que foi desenvolvido neste trabalho. Na primeira seção serão descritos os seus casos de uso e apresentados diagramas de sequência de suas funcionalidades.

Na seção subsequente será feita uma breve descrição das tecnologias utilizadas neste trabalho e por fim será feita uma breve comparação do \textit{plugin} desenvolvido neste trabalho com um já existente e que possui funcionalidades similares.

\section{Arquitetura}

O \textit{plugin} proposto consite de três partes: configurações, sumário de envio e formulário de envio. Nas configurações é onde será definido o comportamento do \textit{plugin}. O sumário de envio será visto pelo estudante e avaliadores na tela inicial da tarefa, e contém um resumo da tarefa. Já o formulário de envio, que é onde o estudante responde a tarefa.

%O plugin desenvolvido neste trabalho permite que o estudante crie um mapa conceitual, modifique sua aparência e envie para avaliação. O estudante poderá também editar um mapa que já foi enviado, além de ter a possibilidade de salva-lo na plataforma CMPaaS. A \autoref{fig_casosuso} apresenta os casos de uso do plugin.
%

Para o desenvolvimento do \textit{plugin} foi elaborado um diagrama de casos de uso, que pode ser visto na \autoref{fig_casosuso}. 

\begin{figure}[!h]
	\caption{\label{fig_casosuso} Casos de Uso}
	\begin{center}
		\includegraphics[scale=0.7]{casosuso.png}
	\end{center}
	\legend{Fonte: Elaborado pelo autor}
\end{figure} 

O \textit{plugin} oferece as seguintes funcionalidades:

\begin{enumerate}
	\item Criar Mapa: permite a criação de um novo mapa conceitual. O novo mapa é também criado nos repositórios do CMPaaS
	\item Salvar Mapa: persiste as alterações do  mapa  criado  nos  repositórios  do  CMPaaS.
	\item  Edição de um mapa previamente salvo no repositório do CMPaaS. Esta funcionalidade é representada pelo caso de uso Editar Mapa.
	\item Visualização de um mapa previamente salvo no repositório do CMPaaS. Esta funcionalidade é representada pelo caso de uso Visualizar Mapa.
	\item Gerenciar os mapas e suas versões armazenados no CMPaaS.
\end{enumerate}  

\subsection{Listar mapas e suas versões}

O diagrama da \autoref{fig_seqlistar} mostra o processo de autenticação e de listagem dos mapas que o estudante possui no CMPaaS. 

\begin{figure}[htb]
	\caption{\label{fig_seqlistar} Listar mapas}
	\begin{center}
		\includegraphics[scale=0.4]{seqlistar.png}
	\end{center}
	\legend{Fonte: Elaborado pelo autor}
\end{figure}

Quando o estudante acessa a atividade uma interface de autenticação é apresentada. O estudante deve então entrar com seus dados de acesso e então efetuar autenticação. Após o estudante ser autenticado a interface solicita ao CMPaaS a lista de mapas que o usuário possui e então uma listagem de mapas é mostrada ao estudante.  

Após a lista de mapas ser apresentada é possível que o estudante visualize as versões que cada mapa possui. O diagrama da \autoref{fig_seqver} mostra o processo de listagem das versões dos mapas existentes no CMPaaS. Para realizar este procedimento o estudante solicita a interface que mostre as versões de um mapa presente em sua lista de mapas conceituais. A interface então faz uma requisição para o CMPaaS solicitando as versões do mapa. Por fim o CMPaaS responde com a lista de versões que então é mostrada ao estudante.   

\begin{figure}[htb]
	\caption{\label{fig_seqver} Listar versões}
	\begin{center}
		\includegraphics[scale=0.4]{seqver.png}
	\end{center}
	\legend{Fonte: Elaborado pelo autor}
\end{figure}



\subsection{Criar e editar mapas}

Quando o estudante executa a funcionalidade de criação de mapas um um editor de mapas conceituais é carregado na interface do \textit{plugin}. Neste editor o estudante cria o mapa desejado e então o armazena no repositório do CMPaaS. Esta sequência de eventos é ilustrada na \autoref{fig_seqcriar}.

\begin{figure}[htb]
	\caption{\label{fig_seqcriar} Criar mapa}
	\begin{center}
		\includegraphics[scale=0.5]{seqcriar.png}
	\end{center}
	\legend{Fonte: Elaborado pelo autor}
\end{figure}

A funcionalidade de edição de mapas funciona de forma semelhante a de criação. A diferença consiste no fato de que, ao editar e salvar o mapa, uma nova versão do mapa que está sendo editado é criado no CMPaaS. Esta sequência de eventos é similar a de criar um mapa e é ilustrada na \autoref{fig_seqeditar}. 

\begin{figure}[htb]
	\caption{\label{fig_seqeditar} Editar mapa}
	\begin{center}
		\includegraphics[scale=0.5]{seqeditar.png}
	\end{center}
	\legend{Fonte: Elaborado pelo autor}
\end{figure}

\subsection{Apagar mapas e suas versões}

Existem duas funcionalidades quer permitem ao estudante apagar de forma definitiva um mapa existente em seu repositório. A primeira, que esta demonstrada na \autoref{fig_apagarver}, consiste em apagar uma única versão de um mapa. Neste caso apenas uma versão é apagada, sendo mantido no repositório o mapa e suas outras versões.
 

\begin{figure}[htb]
	\caption{\label{fig_apagarver} Apagar versão}
	\begin{center}
		\includegraphics[scale=0.5]{apagarver.png}
	\end{center}
	\legend{Fonte: Elaborado pelo autor}
\end{figure}

A outra forma é apagar todas as versões de um mapa, eliminando todo o histórico do mapa do repositório. A sequencia de eventos desta funcionalidade é mostrada na \autoref{fig_apagarmap}. 

\begin{figure}[htb]
	\caption{\label{fig_apagarmap} Apagar mapa}
	\begin{center}
		\includegraphics[scale=0.5]{apagarmap.png}
	\end{center}
	\legend{Fonte: Elaborado pelo autor}
\end{figure}

\section{Tecnologias utilizadas}

\subsection{JavaScript}

O JavaScript é uma linguagem orientada a objeto criada em 1995 por Brendan Eich que foi idealizada para permitir que pessoas que não são programadoras pudessem estender as funcionalidades de sites de Internet \cite{Richards2010}. Ela é uma linguagem interpretada, e apesar de ter sido criada para desenvolvimento web, ela é uma linguagem de proposito geral, e pode ser usada para o desenvolvimento de qualquer tipo de aplicação \cite{flanagan2006}.

Ela foi escolhida para ser utilizada neste projeto pois o editor de mapas disponível no CMPaaS foi desenvolvido com esta tecnologia e, além disso, trata-se de uma linguagem compatível com todos os navegadores modernos.  

\subsection{Json}

O \begin{otherlanguage*}{english}\textit{JavaScript Object Notation}\end{otherlanguage*} (Json) é uma linguagem em formato de texto cuja função é a serialização de dados estruturados. A serialização é o processo de transformar objetos em um fluxo de \textit{bytes} para ser armazenado em um banco de dados ou disco. Ele pode ser utilizado para representar tipos primitivos, como cadeia de caracteres e números, ou tipos estruturados, como objetos e vetores \cite{crockford2006}. Ele é utilizado para intercambio de dados entre os serviços da plataforma CMPaaS.  

\subsection{PHP}
% ---
PHP é um acrônimo recursivo para \begin{otherlanguage*}{english}\textit{PHP: Hypertext Preprocessor}\end{otherlanguage*}, originalmente ele foi criado como uma linguagem de script estruturada cuja finalidade era o desenvolvimento de aplicações com a funcionalidade de geração de HTML dinâmico. Com o passar do tempo a linguagem evoluiu e passou a oferecer recursos para o desenvolvimento orientado a objeto \cite{minetto2007}.

A plataforma Moodle e seus \textit{plugins} são desenvolvidos em PHP, por isto esta linguagem foi utilizada neste projeto. 

\section{Trabalhos correlatos}

Existe um \textit{plugin} semelhante ao desenvolvido neste trabalho no repositório do Moodle. Este \textit{plugin} possibilita que estudantes respondam a questionários com a elaboração de um mapa conceitual. A \autoref{fig_outroplugin} mostra um mapa criado neste \textit{plugin} cuja as principais funcionalidades são a possibilidade de criar um número ilimitado de conceitos e relacionamentos, compatibilidade com padrões de internacionalização e persistência de mapas com a utilização de XML. 

\begin{figure}[htb]
	\caption{\label{fig_outroplugin} Um plugin de elaboração de mapas conceituais}
	\begin{center}
		\includegraphics[scale=0.4]{outroplugin.png}
	\end{center}
	\legend{Fonte: \citeonline{MoodlePlugin2016}}
\end{figure} 


Infelizmente a última versão dele foi lançada em fevereiro de 2015 e não é compatível com as versões mais atuais do Moodle. Um problema que o \textit{plugin} proposto neste trabalho resolve, pois trata-se de uma interface que permite a utilização dos serviços do CMPaaS dentro do Moodle. Esta característica o torna mais independente dos recursos oferecidos pela plataforma, necessitando assim de menor manutenção a cada atualização do Moodle  



\chapter{Desenvolvimento}

Neste capítulo será descrita a implementação da arquitetura que foi proposta no capítulo anterior. Inicialmente serão apresentadas as modificações feitas no editor de mapas presente no CMPaaS com o objetivo de adicionar funcionalidades de formação dos mapas conceituais. Também será descrito como o editor de mapas foi integrado ao \textit{plugin} de envio de tarefas do Moodle, para proporcionar ao estudante a possibilidade de responder a uma tarefa com um mapa conceitual. E por fim será apresentado como foi realizada a integração do \textit{plugin} com o CMPaaS. 

\section{Serviço de Edição de Mapas}
O serviço de edição de mapas foi implementado com a utilização do editor de mapas conceituais disponível no CMPaaS. Ele foi desenvolvido utilizando GoJS, que é uma biblioteca em JavaScript para criação de diagramas para navegadores \textit{Web}.

Este editor salva os mapas no formato Json, possibilitando que outras aplicações utilizem os dados que ele produz. Foi concebido desta forma para acompanhar a proposta do CMPaaS de permitir a interoperabilidade dos serviços oferecidos.

A primeira parte do trabalho foi realizar um aprimoramento deste editor, adicionando funções para a formatação dos mapas criados. Não existia por exemplo, a possibilidade de alterar a fonte da letra dos conceitos ou a cor de fundo dos nós. Uma barra de ferramenta foi adicionada ao editor original para que funções de formatação estivessem disponíveis.

\begin{figure}[htb]
	\caption{\label{fig_barraformacao} Editor com barra de formatação}
	\begin{center}
		\includegraphics[scale=0.5]{barraformacao.png}
	\end{center}
	\legend{Fonte: Elaborada pelo autor}
\end{figure}

% ---
\subsection{Implementação das funções de formatação}
O editor de mapas é dividido em dois componentes, um Model e um Diagram. O Model é o que contém os dados do mapa que está sendo criado, nele os nós e links são descritos por vetores de objetos JavaScript. Já o Diagram é usado para visualizar os dados contidos no modelo.

Na \autoref{fig_gojs} temos um trecho de código que gera um diagrama simples em GoJS. Nele é criado um Model que possui um vetor com três objetos JavaScript, nomeados Alpha, Beta e Gamma. Este vetor de objetos da classe Node representa três nós que serão apresentados no diagrama. O Model nomeado myModel é então adicionado ao Diagram chamado myDiagram. Este trecho de código gera o diagrama ilustrado na \autoref{fig_diagram}.

\begin{figure}[htb]
	\caption{\label{fig_gojs} Componentes do GoJS}
	\begin{center}
		\includegraphics[scale=0.7]{gojs.png}
	\end{center}
	\legend{Fonte: Elaborada pelo autor}
\end{figure}

\begin{figure}[htb]
	\caption{\label{fig_diagram} Um diagrama gerado em GoJS}
	\begin{center}
		\includegraphics[scale=0.7]{diagram.png}
	\end{center}
	\legend{Fonte: Elaborada pelo autor}
\end{figure}

Cada nó do mapa criado pelo editor é representado por um objeto da classe Node, que por sua vez, é composto por blocos que determinam a sua aparência. Os blocos que o editor utiliza são o Shape e o TextBlock. A classe Shape é utilizada para mostrar uma forma geométrica colorida. Já a classe TextBlock tem como função mostrar um texto. Ambas as classes possuem diversas propriedades que servem para determinar a sua aparência e o seu comportamento no diagrama.

Os trechos de código abaixo são usados para criar um Shape e um TextBlock. A forma geométrica  desenhada é um retângulo com largura de 40 pixels, altura de 60 pixels, com margem de 4 pixels e preenchimento na cor vermelha. Já o TextBlock criado é um texto “a Text Block” de cor vermelha.

\begin{figure}[htb]
	\caption{\label{fig_gojselements} Código dos elementos Shape e TextBlock}
	\begin{center}
		\includegraphics[scale=0.6]{gojselements.png}
	\end{center}
	\legend{Fonte: Elaborada pelo autor}
\end{figure}

As novas funcionalidades de formatação adicionadas ao editor modificam as propriedades dos objetos das classes Shape e TextBlock, alterando assim sua aparência. Foram criados seis botões de formatação com as funções de alterar o tipo de fonte, aumentar o tamanho da fonte, alterar o estilo do texto para negrito, itálico e sublinhado, e para alterar a cor de preenchimento dos nós. Os botões de formatação de fonte alteram a propriedade \textit{font} da TextBlock, que deve ser uma \textit{string Cascading Style Sheets} (CSS). Já o botão que altera a cor do nó modifica a propriedade \textit{fill} da Shape.

As funcionalidades de formatação foram implementadas em JavaScript no arquivo toolbar.js. Elas foram desenvolvidas de forma semelhante, utilizando eventos disparados pelos botões da barra de ferramenta.

Para realizar alteração de estilo da fonte e de cor do nó foram implementadas funções que modificam os atributos dos nós. Para cada botão da barra de ferramenta foi criada uma rotina que é executada quando o evento de clique é acionado.

A \autoref{fig_listener} mostra a rotina que altera a cor de preenchimento de um nó. Inicialmente é criada uma espera para o evento que será disparado pelo botão, no caso o evento \textit{input}. Quando ele ocorre é chamada a função que altera a propriedade \textit{fill} do objeto Shape de todos os nós selecionados, mudando assim a cor dos mesmos. As funcionalidades de estilo da fonte são implementadas com a mesma rotina, o que muda é a propriedade alterada, que passa a ser a \textit{font} do objeto TextBlock de todos os nós selecionados.    

\begin{figure}[htb]
	\caption{\label{fig_listener} Rotina que altera a cor de preenchimento de um nó}
	\begin{center}
		\includegraphics[scale=0.6]{listener.png}
	\end{center}
	\legend{Fonte: Elaborada pelo autor}
\end{figure}

\section{Implementação no Moodle} 
Para implementar o serviço de edição de mapas no Moodle foi utilizado um \textit{plugin} já existente na plataforma como modelo. Conforme visto no \autoref{cap-projeto} o \textit{plugin} escolhido para a implementação da funcionalidade de edição de mapas foi o de envio de tarefa. Todos os \textit{plugins} deste tipo possuem uma estrutura de arquivo padrão, que será detalhada abaixo.
\begin{itemize}
	\item version.php: este arquivo contém informações sobre a versão do \textit{plugin}, é utilizado para que o Moodle instale e atualize o \textit{plugins} corretamente.
	\item settings.php: este arquivo permite que se adicione opções personalizadas para a página configuração do \textit{plugins}.
	\item lang/en/submission\_nomedoplugin.php:  este é o arquivo de linguagem, ele é usado para internacionalização do \textit{plugin}.
	\item db/access.php: e utilizada para adicionar capacidades adicionais ao \textit{plugin}. Este arquivo é opcional, não sendo necessário se o \textit{plugin} não tiver capacidades adicionais.
	\item db/upgrade.php: este arquivo define a rotina de atualização do \textit{plugin}. 
	\item db/install.xml: este arquivo define as tabelas de banco de dados que o \textit{plugin} vai utilizar. 
	\item db/install.php: contem o código de instalação do \textit{plugin}.
	\item db/locallib.php: é o arquivo mais importante, é ele que define todas as funcionalidades do \textit{plugin}.    
\end{itemize} 

\subsection{Desenvolvimento do plugin}
O \textit{plugin} utilizado como base para o desenvolvimento deste trabalho foi o de envio de texto online, nele o estudante tem a possibilidade responder uma tarefa por meio de um texto escrito diretamente na plataforma. Este projeto foi desenvolvido aplicando engenharia reversa a este \textit{plugin}, buscando entender a sua funcionalidade e assim descobrir como modificá-lo para transformá-lo em um editor de mapas.

O \textit{plugin} de envio de texto online pode ser dividido em duas partes, uma é o editor de textos onde o estudante realiza a sua tarefa e a outra e o sumário da atividade onde tanto o estudante quanto o avaliador tem acesso ao conteúdo da tarefa enviada. O \textit{plugin} desenvolvido neste trabalho mantém esta estrutura, alterando o editor de textos por um de mapas conceituais e alterando o sumário de forma que o conteúdo apresentado nele passa a ser um Json do mapa ao invés de um texto.

As alterações feitas no \textit{plugin} de envio de texto online se concentraram no arquivo locallib.php, que é o que determina o seu comportamento e funcionalidades. Dentro deste arquivo, as mudanças ocorreram nas funções get\_form\_elements(), que constrói o formulário de envio de tarefa, e no arquivo save(), que realiza a submissão do conteúdo criado pelo estudante.

A primeira parte do trabalho foi substituir o formulário de envio de texto pelo editor de mapas conceituais. Esta modificação foi realizada na função get\_form\_elements(). O código que realizava a inserção do editor de texto foi substituído por um outro, que insere um iframe contendo o editor de mapas do CMPaaS. 

Assim, o \textit{plugin} já apresentava o editor de mapas, porém ainda era necessário salvar o mapa criado no banco de dados do Moodle. Para realizar isto, Json do editor contido no iframe precisava ser salvo em algum campo de formulário. Como solução, foi criado uma entrada de dados, oculta na página da tarefa, responsável por receber o Json do mapa criado no editor.

Quando o estudante aciona o botão de submissão da tarefa, o Json respectivo ao mapa que criado é salvo em um campo de formulário e então armazenado no banco de dados do Moodle.

Para finalizar, foi necessário criar uma forma de o editor de mapas carregar o mapa salvo pelo estudante, para caso ele tivesse interesse em editar um mapa já armazenado. A solução para isto foi realizar o caminho inverso que foi feito ao salvar o mapa criado. Ao carregar o formulário de edição de mapas o Json armazenado no banco de dados e carregado em um campo oculto e um código JavaScript se encarrega se obter o conteúdo dele e carregar no mapa.

A primeira parte do desenvolvimento do \textit{plugin} foi concluída, o estudante conseguia criar um mapa no editor, salvar o seu conteúdo na plataforma e editar novamente caso fosse necessário. Ainda havia a necessidade de permitir que o avaliador tivesse acesso ao conteúdo gerado pelo aluno. Além disso, falta a integração com o CMPaaS e utilização de seus recursos.

\section{Integração com o CMPaaS}

A integração com o CMPaaS é funcionalidade mais importante do \textit{plugin}, visto que o objetivo do trabalho é aumentar a base de usuários e o conteúdo do CMPaaS.

O \textit{plugin} utiliza a API Rest do CMPaaS para guardar mapas no repositório da plataforma e também para recuperar mapas que foram previamente armazenados nela pelo usuário. 

\subsection{Lista de mapas do usuário}

Para obter a lista de mapas do usuário o \textit{plugin} envia um request de método GET com URI /api/users/. A API então responde com um JSON com os dados do usuário e sua lista de mapas, conforme visto na \autoref{fig_user_maps}. Este JSON contém a URI do usuário, seu \textit{username} e também a lista de URIs dos mapas que o usuário possui na plataforma.

\begin{figure}[htb]
	\caption{\label{fig_user_maps} Json com dados do usuário}
	\begin{center}
		\includegraphics[scale=0.6]{user_maps.png}
	\end{center}
	\legend{Fonte: Elaborada pelo autor}
\end{figure}

A partir deste JSON o plugin elabora novas requests que serão enviadas para a API do CMPaaS, para que sejam obtidas as informações dos mapas do usuário. Para cada URI de mapa que o usuário possui no repositório é feita uma nova request GET para URI com formato /api/maps/id de forma a ser obtida uma resposta contendo um JSON com metadados de cada mapa. Este JSON de resposta é representada na \autoref{fig_map_data}.

\begin{figure}[htb]
	\caption{\label{fig_map_data} Json com os metadados de uma mapa}
	\begin{center}
		\includegraphics[scale=0.6]{map_data.png}
	\end{center}
	\legend{Fonte: Elaborada pelo autor}
\end{figure}

Os metadados do mapa contidos neste JSON são a sua URI, o autor, o título, a questão e a sua descrição. Com estas informações é criada a interface que mostra a listagem de mapas do usuário, conforme \autoref{fig_map_list}. Nesta interface o usuário pode visualizar as versões de um mapa listado ou apagar este mapa.

\begin{figure}[htb]
	\caption{\label{fig_map_list} Lista de mapas de um usuário}
	\begin{center}
		\includegraphics[scale=0.2]{map_list.png}
	\end{center}
	\legend{Fonte: Elaborada pelo autor}
\end{figure}

\subsection{Lista de versões de um mapa}      

O repositório de mapas do CMPaaS possui a funcionalidade de controle de versão, assim sempre que um mapa é editado uma nova versão é criada e a antiga é mantida. O \textit{plugin} utiliza esta funcionalidade de versionamento, permitindo que o usuário visualize as versões de um mapa e as manipule. Esta interface pode ser vista na \autoref{fig_map_versions}.

\begin{figure}[htb]
	\caption{\label{fig_map_versions} Versões de um mapa}
	\begin{center}
		\includegraphics[scale=0.4]{map_versions.png}
	\end{center}
	\legend{Fonte: Elaborada pelo autor}
\end{figure}

O conteúdo desta interface é obtido através de uma request GET enviada para a URI /api/maps/id/version. A API responde a esta request com um JSON contendo as versões do mapa solicitado, além dos metadados e conteúdo do mapa conceitual de cada versão. A resposta obtida pode ser vista na \autoref{fig_version_list}. 

\begin{figure}[htb]
	\caption{\label{fig_version_list} Json com as versões de um mapa}
	\begin{center}
		\includegraphics[scale=0.6]{version_list.png}
	\end{center}
	\legend{Fonte: Elaborada pelo autor}
\end{figure} 

Este Json contém a lista de versões de um mapa e o conteúdo de cada uma. Os dados de cada versão são seu identificador, a data e horário de criação, a URI do mapa a qual ela pertence, seu autor e o conteúdo da versão, que é a representação do diagrama do mapa conceitual criado pelo usuário. Este conteúdo da versão é o Json que será utilizado pelo editor de mapas para desenhar o diagrama criado pelo usuário.

\subsection{O editor de mapas}

Quando o usuário aciona o botão de editar uma versão de um mapa a função loadMapVersionContent() é chamada. Esta função armazena no localStorage do browser a URI do mapa e o Json do conteúdo que representa o diagrama do mapa conceitual. Após os dados serem salvos no localStorage a função carrega a interface do editor de mapas. A URI do mapa então é utilizada em uma request GET, ilustrado na \autoref{fig_map_data}, para carregar metadados na interface do editor e o diagrama é finalmente carregado e plotado para o usuário editar. O resultado final pode ser visto na \autoref{fig_editor}.

\begin{figure}[htb]
	\caption{\label{fig_editor} Editor de mapas}
	\begin{center}
		\includegraphics[scale=0.2]{editor.png}
	\end{center}
	\legend{Fonte: Elaborada pelo autor}
\end{figure}

\subsection{Exclusão de mapas} 

O \textit{plugin} também possui a funcionalidade de exclusão de um mapa e as suas versões. As exclusões são realizadas através do envio de requests do tipo DELETE para a API do CMPaaS. Uma versão é apagada com o envio de uma request DELETE para a URI api/maps/id/versions/id/. Já para excluir um mapa é necessário enviar a mesma request para a URI api/maps/id/versions/id/.   

\chapter{Instalação e Testes}

\section{Instalação}
A pasta contendo os arquivos do \textit{plugin} deve ser copiada para o diretório do Moodle mod/assign/submission/. Após copiar os arquivos para este local é necessário acessar o Moodle com um usuário com perfil de administrador. Assim que efetuar login irá aparecer uma mensagem de instalação de \textit{plugin}, conforme a \autoref{fig_installplugin}.
\begin{figure}[htb]
	\caption{\label{fig_installplugin} Configurando a tarefa para utilizar o \textit{plugin}}
	\begin{center}
		\includegraphics[scale=0.4]{installplugin.png}
	\end{center}
	\legend{Fonte: Elaborada pelo autor}
\end{figure}

Ao clicar no botão de atualização do banco de dados do Moodle o \textit{plugin} será instalado e poderá ser visto na lista de \textit{plugins}, conforme \autoref{fig_listaplugin}.

\begin{figure}[htb]
	\caption{\label{fig_listaplugin} Lista de \textit{plugins} instalados}
	\begin{center}
		\includegraphics[scale=0.4]{listaplugin.png}
	\end{center}
	\legend{Fonte: Elaborada pelo autor}
\end{figure}

Depois do \textit{plugin} ser instalado é necessário realizar a configuração da tarefa para utilizá-lo. Para que a tarefa utilize o \textit{plugin} para submissão de dados é necessário selecioná-lo na tela de configuração da atividade, conforme ilustrado na \autoref{fig_configplugin}.

\begin{figure}[htb]
	\caption{\label{fig_configplugin} Configurando a tarefa para utilizar o plugin}
	\begin{center}
		\includegraphics[scale=0.4]{configplugin.png}
	\end{center}
	\legend{Fonte: Elaborada pelo autor}
\end{figure}

\begin{figure}[htb]
	\caption{\label{fig_editor2} Editor de mapas}
	\begin{center}
		\includegraphics[scale=0.2]{editor.png}
	\end{center}
	\legend{Fonte: Elaborada pelo autor}
\end{figure}

\section{Como utilizar}

A utilização do \textit{plugin} é bem simples. Ao acessar uma tarefa o estudante deve clicar no botão de envio de tarefa e ao fazer isto será mostrada uma tela de login, onde o usuário deverar realizar autenticação no CMPaaS. Esta tela é mostrada na \autoref{fig_login}. Após o login ser realizado o usuário é redirecionado para a interface que contém a lista de mapas que ele possui armazenado no CMPaaS. Esta lista foi apresentada anteriormenta na \autoref{fig_map_list}. 

\begin{figure}[htb]
	\caption{\label{fig_login} Interface de login}
	\begin{center}
		\includegraphics[scale=0.5]{login.png}
	\end{center}
	\legend{Fonte: Elaborada pelo autor}
\end{figure}

O estudante então pode criar um novo mapa ou então editar mapas já existentes. Após criar ou editar o mapa o aluno deve clicar no botão “Salvar alterações” e o Json será salvo no banco de dados da plataforma e aparecerá no sumário, conforme \autoref{fig_sumario}. 

\begin{figure}[htb]
	\caption{\label{fig_sumario} Sumário da tarefa}
	\begin{center}
		\includegraphics[scale=0.3]{sumario.png}
	\end{center}
	\legend{Fonte: Elaborada pelo autor}
\end{figure}

\section{Provas de Conceitos}

Nesta seção será apresentada a utilização de todas as funcionalidades oferecidas pelo \textit{plugin}. Elas podem ser dividas em dois grupos: recursos para elaboração e edição dos mapas e persistência dos diagramas.

Os recursos para edição de mapas são as funções de formatação da aparência do diagrama disponíveis na barra de ferramentas. Elas permitem alterar a cor de preenchimento dos nós e o estilo da fonte de cada conceito. A \autoref{fig_mapa} mostra um mapa no qual foi aplicado estas funções.

Já a persistência do diagrama pode ser dividida em duas etapas: a gravação do mapa no Moodle e a persistencia dos mapas no repositório do CMPaaS. A primeira ocorre quando o usuário clica no botão Salvar Alterações, que foi apresentado na \autoref{fig_editor}. Ao fazer isto, o Json do mapa que está sendo editado é salvo no banco de dados do Moodle.
 
Já a persistencia no CMPaaS ocorre quando o estudante utiliza o botão Salvar Mapa Conceitual, que pode ser visto na cor azul na \autoref{fig_mapa}.

\begin{figure}[htb]
	\caption{\label{fig_mapa} Mapa que utiliza os recursos de formação}
	\begin{center}
		\includegraphics[scale=0.3]{mapa.png}
	\end{center}
	\legend{Fonte: Elaborada pelo autor}
\end{figure}


% ----------------------------------------------------------
% Finaliza a parte no bookmark do PDF
% para que se inicie o bookmark na raiz
% e adiciona espaço de parte no Sumário
% ----------------------------------------------------------
\phantompart

% ---
% Conclusão
% ---
\chapter{Considerações finais}

\section{Resultados}

O objetivo deste trabalho foi a criação de uma ferramenta que permitisse a utilização de mapas conceituais em cursos gerenciados pela plataforma Moodle. O \textit{plugin}  deveria também possuir integração com o CMPaaS. Para realizar isto foi idealizado o desenvolvimento de um \textit{plugin} de envio de tarefa que oferecesse a possibilidade do estudante responder uma atividade com a elaboração de um mapa conceitual.

O protótipo desenvolvido neste trabalho alcançou o objetivo de oferecer uma ferramenta para o estudante responder uma tarefa com mapas conceituais. O \textit{plugin} desenvolvido possui um editor de mapas que permite a elaboração de diagramas e modificação de sua aparência, além de realizar a persistência dos mapas criados. 

A integração com o CMPaaS também foi implementada, assim o estudante tem a possibilidade de armazenar o mapa criado no editor com o serviço de armazenamento na Nuvem oferecido pela plataforma. É possível também manipular mapas criados anteriormente. 



\section{Conclusão}

Neste trabalho foi apresentada a definição de mapas conceituais e a sua importância para a educação. Neste contexto foi discutido como eles podem ser utilizados no ensino e foi levantado quais são as ferramentas disponíveis para criação e manipulação de mapas. Neste levantamento foi identificada uma carência de aplicações computacionais que proporcionam suporte a criação de mapas conceituais e a necessidade do desenvolvimento de novas ferramentas com esta finalidade.

Foi pensado então o projeto de uma ferramenta que proporcionasse a aplicação de mapas conceituais na Educação a Distância. Escolhemos o Moodle como a plataforma onde seria realizada a implementação deste projeto. Ele foi escolhido por ser um gerenciador de cursos open-source amplamente utilizado em todo o mundo e por ser facilmente extensível através de inclusão de novos módulos.

Um \textit{plugin} que permite a aplicação de mapas conceituais em tarefas dos cursos do Moodle foi então desenvolvido. Este \textit{plugin} possibilita que os estudantes respondam tarefas com a elaboração de mapas conceituais e oferece a possibilidade de sincronização do mapa na plataforma CMPaaS. 
    

\section{Trabalhos futuros}
% ---


O \textit{plugin} que foi criado neste trabalho trata-se de um protótipo, ou seja, não está totalmente funcional e carece de melhorias. No entanto serve como base para a criação de uma ferramenta que pode ser adicionada a lista de repositórios de \textit{plugin} do Moodle, ampliando assim a possibilidade do uso de mapas conceituais no ensino.     

O protótipo desenvolvido neste trabalho teve foco na interface para o estudante criar mapas conceituais, não possuindo o recurso de apresentar o mapa elaborado nele para um avaliador. Assim, uma proposta de trabalho futuro é a criação de uma forma de o avaliador visualizar o mapa submetido pelo estudante.

Ademais é sugerido o desenvolvimento de trabalhos que visam a melhoria da experiência do usuário que utiliza o editor de mapas. Visto que este trabalho não teve como foco a criação de uma interface otimizada.

% ----------------------------------------------------------
% ELEMENTOS PÓS-TEXTUAIS
% ----------------------------------------------------------
\postextual
% ----------------------------------------------------------

% ----------------------------------------------------------
% Referências bibliográficas
% ----------------------------------------------------------
\bibliography{abntex2-modelo-references}

% ----------------------------------------------------------
% Glossário
% ----------------------------------------------------------
%
% Consulte o manual da classe abntex2 para orientações sobre o glossário.
%
%\glossary

% ----------------------------------------------------------
% Apêndices
% ----------------------------------------------------------

% ---
% Inicia os apêndices
% ---
%\begin{apendicesenv}

% Imprime uma página indicando o início dos apêndices
%\partapendices


%\end{apendicesenv}
% ---


% ----------------------------------------------------------
% Anexos
% ----------------------------------------------------------

% ---
% Inicia os anexos
% ---
%\begin{anexosenv}

% Imprime uma página indicando o início dos anexos
%\partanexos


%\end{anexosenv}

%---------------------------------------------------------------------
% INDICE REMISSIVO
%---------------------------------------------------------------------
%\phantompart
%\printindex
%---------------------------------------------------------------------

\end{document}
