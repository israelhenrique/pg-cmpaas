%% abtex2-modelo-trabalho-academico.tex, v-1.9.6 laurocesar
%% Copyright 2012-2016 by abnTeX2 group at http://www.abntex.net.br/ 
%%
%% This work may be distributed and/or modified under the
%% conditions of the LaTeX Project Public License, either version 1.3
%% of this license or (at your option) any later version.
%% The latest version of this license is in
%%   http://www.latex-project.org/lppl.txt
%% and version 1.3 or later is part of all distributions of LaTeX
%% version 2005/12/01 or later.
%%
%% This work has the LPPL maintenance status `maintained'.
%% 
%% The Current Maintainer of this work is the abnTeX2 team, led
%% by Lauro César Araujo. Further information are available on 
%% http://www.abntex.net.br/
%%
%% This work consists of the files abntex2-modelo-trabalho-academico.tex,
%% abntex2-modelo-include-comandos and abntex2-modelo-references.bib
%%

% ------------------------------------------------------------------------
% ------------------------------------------------------------------------
% abnTeX2: Modelo de Trabalho Academico (tese de doutorado, dissertacao de
% mestrado e trabalhos monograficos em geral) em conformidade com 
% ABNT NBR 14724:2011: Informacao e documentacao - Trabalhos academicos -
% Apresentacao
% ------------------------------------------------------------------------
% ------------------------------------------------------------------------

\documentclass[
	% -- opções da classe memoir --
	12pt,				% tamanho da fonte
	openright,			% capítulos começam em pág ímpar (insere página vazia caso preciso)
	oneside,			% para impressão em recto e verso. Oposto a oneside
	a4paper,			% tamanho do papel. 
	% -- opções da classe abntex2 --
	%chapter=TITLE,		% títulos de capítulos convertidos em letras maiúsculas
	%section=TITLE,		% títulos de seções convertidos em letras maiúsculas
	%subsection=TITLE,	% títulos de subseções convertidos em letras maiúsculas
	%subsubsection=TITLE,% títulos de subsubseções convertidos em letras maiúsculas
	% -- opções do pacote babel --
	english,			% idioma adicional para hifenização
	french,				% idioma adicional para hifenização
	spanish,			% idioma adicional para hifenização
	brazil				% o último idioma é o principal do documento
	]{abntex2}

% ---
% Pacotes básicos 
% ---
\usepackage{lmodern}			% Usa a fonte Latin Modern			
\usepackage[T1]{fontenc}		% Selecao de codigos de fonte.
\usepackage[utf8]{inputenc}		% Codificacao do documento (conversão automática dos acentos)
\usepackage{lastpage}			% Usado pela Ficha catalográfica
\usepackage{indentfirst}		% Indenta o primeiro parágrafo de cada seção.
\usepackage{color}				% Controle das cores
\usepackage{graphicx}			% Inclusão de gráficos
\usepackage{microtype} 			% para melhorias de justificação
% ---
		
% ---
% Pacotes adicionais, usados apenas no âmbito do Modelo Canônico do abnteX2
% ---
\usepackage{lipsum}				% para geração de dummy text
% ---

% ---
% Pacotes de citações
% ---
%\usepackage[brazilian,hyperpageref]{backref}	 % Paginas com as citações na bibl
\usepackage[alf]{abntex2cite}	% Citações padrão ABNT

% --- 
% CONFIGURAÇÕES DE PACOTES
% --- 

% ---
% Configurações do pacote backref
% Usado sem a opção hyperpageref de backref
%\renewcommand{\backrefpagesname}{Citado na(s) página(s):~}
% Texto padrão antes do número das páginas
%\renewcommand{\backref}{}
% Define os textos da citação
%\renewcommand*{\backrefalt}[4]{
%	\ifcase #1 %
%		Nenhuma citação no texto.%
%	\or
%		Citado na página #2.%
%	\else
%		Citado #1 vezes nas páginas #2.%
%	\fi}%
% ---

% ---
% Informações de dados para CAPA e FOLHA DE ROSTO
% ---
\titulo{Um plugin de mapas conceituais para o Moodle}
\autor{Israel Henrique Silva de Lima}
\local{Vitória}
\data{2016}
\orientador{Wagner de Andrade  Perin}
\coorientador{Davidson Cury}
\instituicao{
	Universidade Federal do Espírito Santo -- UFES
	\par
	Centro Tecnológico
	\par
	Departamento de Informática}
\tipotrabalho{Monografia (PG)}
% O preambulo deve conter o tipo do trabalho, o objetivo, 
% o nome da instituição e a área de concentração 
\preambulo{Monografia apresentada ao Curso de Engenharia de Computação do Departamento de Informática da Universidade Federal do Espírito Santo, como requisito parcial para obtenção do Grau de Bacharel em Engenharia de Computação.}
% ---


% ---
% Configurações de aparência do PDF final

% alterando o aspecto da cor azul
\definecolor{blue}{RGB}{41,5,195}

% informações do PDF
\makeatletter
\hypersetup{
     	%pagebackref=true,
		pdftitle={\@title}, 
		pdfauthor={\@author},
    	pdfsubject={\imprimirpreambulo},
	    pdfcreator={LaTeX with abnTeX2},
		pdfkeywords={abnt}{latex}{abntex}{abntex2}{trabalho acadêmico}, 
		colorlinks=true,       		% false: boxed links; true: colored links
    	linkcolor=blue,          	% color of internal links
    	citecolor=blue,        		% color of links to bibliography
    	filecolor=magenta,      		% color of file links
		urlcolor=blue,
		bookmarksdepth=4
}
\makeatother
% --- 

% --- 
% Espaçamentos entre linhas e parágrafos 
% --- 

% O tamanho do parágrafo é dado por:
\setlength{\parindent}{1.3cm}

% Controle do espaçamento entre um parágrafo e outro:
\setlength{\parskip}{0.2cm}  % tente também \onelineskip

% ---
% compila o indice
% ---
\makeindex
% ---

% ----
% Início do documento
% ----
\begin{document}

% Seleciona o idioma do documento (conforme pacotes do babel)
%\selectlanguage{english}
\selectlanguage{brazil}

% Retira espaço extra obsoleto entre as frases.
\frenchspacing 

% ----------------------------------------------------------
% ELEMENTOS PRÉ-TEXTUAIS
% ----------------------------------------------------------
% \pretextual

% ---
% Capa
% ---
\imprimircapa
% ---

% ---
% Folha de rosto
% (o * indica que haverá a ficha bibliográfica)
% ---
\imprimirfolhaderosto*
% ---

% ---
% Inserir a ficha bibliografica
% ---

% Isto é um exemplo de Ficha Catalográfica, ou ``Dados internacionais de
% catalogação-na-publicação''. Você pode utilizar este modelo como referência. 
% Porém, provavelmente a biblioteca da sua universidade lhe fornecerá um PDF
% com a ficha catalográfica definitiva após a defesa do trabalho. Quando estiver
% com o documento, salve-o como PDF no diretório do seu projeto e substitua todo
% o conteúdo de implementação deste arquivo pelo comando abaixo:
%
% \begin{fichacatalografica}
%     \includepdf{fig_ficha_catalografica.pdf}
% \end{fichacatalografica}

%\begin{fichacatalografica}
%	\sffamily
%	\vspace*{\fill}					% Posição vertical
%	\begin{center}					% Minipage Centralizado
%	\fbox{\begin{minipage}[c][8cm]{13.5cm}		% Largura
%	\small
%	\imprimirautor
%	%Sobrenome, Nome do autor
%	
%	\hspace{0.5cm} \imprimirtitulo  / \imprimirautor. --
%	\imprimirlocal, \imprimirdata-
%	
%	\hspace{0.5cm} \pageref{LastPage} p. : il. (algumas color.) ; 30 cm.\\
%	
%	\hspace{0.5cm} \imprimirorientadorRotulo~\imprimirorientador\\
%	
%	\hspace{0.5cm}
%	\parbox[t]{\textwidth}{\imprimirtipotrabalho~--~\imprimirinstituicao,
%	\imprimirdata.}\\
%	
%	\hspace{0.5cm}
%		1. Palavra-chave1.
%		2. Palavra-chave2.
%		2. Palavra-chave3.
%		II. Universidade Federal do Espírito Santo.
%		III. Faculdade de xxx.
%		IV. \imprimirtitulo 			
%	\end{minipage}}
%	\end{center}
%\end{fichacatalografica}
% ---



% ---
% Inserir folha de aprovação
% ---

% Isto é um exemplo de Folha de aprovação, elemento obrigatório da NBR
% 14724/2011 (seção 4.2.1.3). Você pode utilizar este modelo até a aprovação
% do trabalho. Após isso, substitua todo o conteúdo deste arquivo por uma
% imagem da página assinada pela banca com o comando abaixo:
%
% \includepdf{folhadeaprovacao_final.pdf}
%
\begin{folhadeaprovacao}

  \begin{center}
    {\ABNTEXchapterfont\large\imprimirautor}

    \vspace*{\fill}\vspace*{\fill}
    \begin{center}
      \ABNTEXchapterfont\bfseries\Large\imprimirtitulo
    \end{center}
    \vspace*{\fill}
    
    \hspace{.45\textwidth}
    \begin{minipage}{.5\textwidth}
        \imprimirpreambulo
    \end{minipage}%
    \vspace*{\fill}
   \end{center}
        
   Trabalho aprovado. \imprimirlocal, 24 de novembro de 2012:

   \assinatura{\textbf{\imprimirorientador} \\ Orientador} 
   \assinatura{\textbf{Professor} \\ Convidado 1}
   \assinatura{\textbf{Professor} \\ Convidado 2}
   %\assinatura{\textbf{Professor} \\ Convidado 3}
   %\assinatura{\textbf{Professor} \\ Convidado 4}
      
   \begin{center}
    \vspace*{0.5cm}
    {\large\imprimirlocal}
    \par
    {\large\imprimirdata}
    \vspace*{1cm}
  \end{center}
  
\end{folhadeaprovacao}
% ---

% ---
% Dedicatória
% ---
% \begin{dedicatoria}
%    \vspace*{\fill}
%    \centering
%    \noindent
%    \textit{ Este trabalho é dedicado às crianças adultas que,\\
%    quando pequenas, sonharam em se tornar cientistas.} \vspace*{\fill}
% \end{dedicatoria}
% ---

% ---
% Agradecimentos
% ---
% \begin{agradecimentos}
% \end{agradecimentos}
% ---

% ---
% Epígrafe
% ---
%\begin{epigrafe}
%     \vspace*{\fill}
% 	\begin{flushright}
% 		\textit{``Não vos amoldeis às estruturas deste mundo, \\
% 		mas transformai-vos pela renovação da mente, \\
% 		a fim de distinguir qual é a vontade de Deus: \\
% 		o que é bom, o que Lhe é agradável, o que é perfeito.\\
% 		(Bíblia Sagrada, Romanos 12, 2)}
% 	\end{flushright}
% \end{epigrafe}
% ---

% ---
% RESUMOS
% ---

% resumo em português
\setlength{\absparsep}{18pt} % ajusta o espaçamento dos parágrafos do resumo
\begin{resumo}
 Segundo a \citeonline[3.1-3.2]{NBR6028:2003}, o resumo deve ressaltar o
 objetivo, o método, os resultados e as conclusões do documento. A ordem e a extensão
 destes itens dependem do tipo de resumo (informativo ou indicativo) e do
 tratamento que cada item recebe no documento original. O resumo deve ser
 precedido da referência do documento, com exceção do resumo inserido no
 próprio documento. (\ldots) As palavras-chave devem figurar logo abaixo do
 resumo, antecedidas da expressão Palavras-chave:, separadas entre si por
 ponto e finalizadas também por ponto.

 \textbf{Palavras-chave}: latex. abntex. editoração de texto.
\end{resumo}

% resumo em inglês
% \begin{resumo}[Abstract]
%  \begin{otherlanguage*}{english}
%    This is the english abstract.
% 
%    \vspace{\onelineskip}
% 
%    \noindent 
%    \textbf{Keywords}: latex. abntex. text editoration.
%  \end{otherlanguage*}
% \end{resumo}


% ---
% inserir lista de ilustrações
% ---
\pdfbookmark[0]{\listfigurename}{lof}
\listoffigures*
\cleardoublepage
% ---

% ---
% inserir lista de tabelas
% ---
%\pdfbookmark[0]{\listtablename}{lot}
%\listoftables*
%\cleardoublepage
% ---

% ---
% inserir lista de abreviaturas e siglas
% ---
%\begin{siglas}
%  \item[ABNT] Associação Brasileira de Normas Técnicas
%  \item[abnTeX] ABsurdas Normas para TeX
%\end{siglas}
% ---

% ---
% inserir lista de símbolos
% ---
%\begin{simbolos}
%  \item[$ \Gamma $] Letra grega Gama
%  \item[$ \Lambda $] Lambda
%  \item[$ \zeta $] Letra grega minúscula zeta
%  \item[$ \in $] Pertence
%\end{simbolos}
% ---

% ---
% inserir o sumario
% ---
\pdfbookmark[0]{\contentsname}{toc}
\tableofcontents*
\cleardoublepage
% ---



% ----------------------------------------------------------
% ELEMENTOS TEXTUAIS
% ----------------------------------------------------------
\textual

% ----------------------------------------------------------
% Introdução (exemplo de capítulo sem numeração, mas presente no Sumário)
% ----------------------------------------------------------
\chapter{Introdução}
%\addcontentsline{toc}{chapter}{Introdução}
% ----------------------------------------------------------


% ----------------------------------------------------------
% PARTE
% ----------------------------------------------------------
%\part{Preparação da pesquisa}
% ----------------------------------------------------------

% ---
% Capitulo com exemplos de comandos inseridos de arquivo externo 
% ---
%\include{abntex2-modelo-include-comandos}
% ---

% ----------------------------------------------------------
% PARTE
% ----------------------------------------------------------
%\part{Referenciais teóricos}
% ----------------------------------------------------------

% ---
% Capitulo de revisão de literatura
% ---
\chapter{Mapas conceituais}
% ---
%O que são? Quem criou? Quando surgiu? Quais os benefícios para a educação?
% ---
\section{Origem e definição de mapas conceituais}
% ---
O mapas conceituais foram criados na década de 70 durante um estudo de Novak que buscava entender como crianças aprendiam conceitos científicos \cite{Novak2005}. Durante seu estudo, Novak observou uma evolução nas proposições utilizadas pelas crianças, porém teve dificuldade de entender como esta mudança congnitiva ocorreu. Para compreender as mudanças na estruturas cognitivas observadas Novak desenvolveu um método baseado nas três ideias de Ausubel sobre a teoria da aprendizagem \cite{ausubel1963}. A primeira ideia é que novos significados são criados a partir de conceitos e proposições previamente conhecidos pelo aprendiz. A segunda, é que a estrutura cognitiva do indivíduo é organizada hierarquicamente, com conceitos mais gerais ocupando posições superiores, acima dos conceitos mais específicos e complexos. Já a terceira idéia é que durante o processo de aprendizagem os conceitos vão se tornando mais específicos e precisos.

\begin{citacao}[english]
	First, Ausubel sees the development
	of new meanings as building on prior relevant concepts and propositions. Second,
	he sees cognitive structure as organised hierarchically, with more general, more inclusive concepts occupying higher levels in the hierarchy and more specific, less
	inclusive concepts subsumed under the more general concepts. Third, when meaningful learning occurs, relationships between concepts become more explicit, more
	precise, and better integrated with other concepts and propositions\cite{Novak2005}.
\end{citacao}

Assim foi criada essa ferramenta gráfica, que tem como objetivo fazer uma representação do conhecimento. Um mapa conceitual é constituido por interligações entre dois ou mais conceitos através de uma palavra de forma a gerar um significado. Ele é construido ligando-se um conceito a outro, através de um arco que contém uma palavra descritiva, que é usada para criar uma relação significativa entre os conceitos. Além disso, os conceitos mais gerais estão dispostos no topo do mapa e os mais específicos na parte inferior dele. A \autoref{fig_mapconceitual} mostra um exemplo de mapa conceitual.

\begin{figure}[htb]
	\begin{center}
		\includegraphics[scale=0.3]{mapaconceitual.png}
	\end{center}
	\caption{\label{fig_mapconceitual}Um exemplo de mapa conceitual Fonte: \cite[p. 28]{Perin2014}.}
\end{figure}
\section{Mapas conceituais e educação}
Os mapas conceituais podem ter diversas aplicações na educação, podem ser utilizados para representar o conhecimento adquirido em uma aula, ou então o conteúdo de um capítulo algum livro lido. Por se tratar de uma representação que é alimentada com novos conceitos conforme eles são adquiridos, pode ser utilizado pelo professor para acompanhar a aprendizagem de um estudante durante todo o período de um curso. Além disso, como os mapas conseguem representar a estrutura cognitiva de um indivíduo, eles podem ser utilizados também como método de avaliação de aprendizagem\cite{Perin2014}.

A confiabilidade da avaliação da aprendizagem tradicional,como textos dissertativos e questões de multipla escolha, tem sido questionada por pesquisadores e educadores. Nestes tipos de teste apenas os acertos são contabilizados e os erros são descartados, isto pode fazer com que informações importantes para a avaliação sejam desprezadas.
As provas tradicionais não proporcionam a possibilidade do estudante apresentar como construiu o seu aprendizado. Elas conseguem avaliar somente a aprendizagem mecânica, não mostrando como que o aprendiz alterou suas estruturas cognitivas.
Os mapas conceituais tem se destacado como alternativa a avaliação tradicional, pois conseguem demonstrar com facilidade as modificações cognitivas que ocorrem durante o processo de aprendizagem do estudante\cite{Dutra2002}.

\section{Mapas conceituais e tecnologia da informação}

A construção de mapas conceituais é muito simples, uma caneta e um pedaço de papel são ferramentas suficientes para a confecção desse grafo. Porém a tarefa de revisa-lo, armazena-lo, e edita-lo a longo prazo  pode ser muito cansativa e complexa. A introdução do uso de computadores na confecção de mapas pode facilitar a tarefa \cite{Novak2006}.

Os primeiros programas de computadores criados com este objetivo se limitavam a mostrar os mapas na tela, sem oferecer nenhum recurso adicional a ferramenta. Com a popularização do uso de computadores e da Internet houve um aumento na oferta de aplicações que tem como objetivo a construção de mapas gráficos. Porém há uma grande confusão entre usuários e desenvolvedores sobre a diferença entre mapas conceituais, organogramas e mapas mentais\cite{Perin2014}. Segundo a pesquisa de \citeonline{Perin2014}, apenas o software CmapTools pode ser considerada uma ferramenta que oferece suporte a criação de mapas conceituais.

\begin{citacao}[english]
	CmapTools  is  a  software  environment  developed  at  the  Institute for Human and Machine Cognition (IHMC) that empowers users, individually or  collaboratively, to represent their knowledge using concept maps, to share them with peers and colleagues, and to publish them\cite{Canas2004}.
\end{citacao}

Como grande parte das ferramentas para edição de mapas gráficos não foram desenvolvidas especificamente para prover um ambiente de criação de mapas conceituais é possível levantar a hipótese de que há uma carência deste tipo de software. Isto é um elemento motivador para a realização deste trabalho.
  

%\lipsum[1]

%\lipsum[2-3]

% ----------------------------------------------------------
% PARTE
% ----------------------------------------------------------
%\part{Resultados}
% ----------------------------------------------------------

% ---
% primeiro capitulo de Resultados
% ---
\chapter{Educação a distância}
% ---
%O que é a educação a distância?
%Qual é a origem da educação a distância?
%Qual é a importancia da educação a distância?
%Quais são as ferramentas de educação a distância?
%O Moodle
%Mapas conceituais e educação a distância.
% ---
\section{Definição da educação a distância}
\section{Origem da educação a distância}
A educação a distância, ao contrario do que se pensa, surgiu muito antes da existência da Internet, porém a sua origem exata é alvo de controvérsias. Existem compêndios que apontam as epístolas de São Paulo a comunidades cristãs da Àsia Menor como a origem da educação a distância. Estas epístolas foram enviadas no século I e ensinavam as comunidades a viver segundo a doutrina cristã\cite{gouvea2006}.

\citeonline{Sherry1996} aponta os cursos por correspondência europeus como a primeira forma de educação a distância.
Segundo \citeonline{landim1997} inicialmente a educação a distância fazia parte de iniciativas isoladas de alguns educadores, sendo somente instituicionalizada na segunda metade do século XIX.

Alguns acontecimentos importante na história da EAD foram a criação da primeira escola de línguas por correspondência em Berlin, no ano de 1856; em 1892 foi criada a Divisão de Ensino por Correspondência na Univeridade de Chicago; o nascimento da Tele Escola Primária do Ministério da Cultura e Educação, na Argentina em 1960; entre outros\cite{Alves2011}.

A evolução da EAD pode ser resumida em cinco gerações, de acordo com as tecnologias utilizadas. A primeira geração seria a educação por correspondência. Já a segunda, ocorre na década de 60 e se baseia no uso do rádio e televisão. A terceira, a partir dos anos 80, é representada por softwares multimídia e acesso a Internet como forma de obtenção de conhecimento. A quarta geração consiste no uso de ambientes virtuais de aprendizagem onde a interação entre professores e estudantes é realizada através da Internet. A quinte e última é representada por ambientes de realidade virtual\cite{maia2003}.

Atualmente a educação a distância está relacionada ao uso de um AVA, que consiste em um sistema computacional cuja finalidade é prover suporte a atividades mediadas pelas tecnologias de informação e comunicação\cite{Almeida2010}. Um ambiente virtual de aprendizagem tem como objetivo integrar mídias e recursos, além de apresentar informações de forma organizada e permitir interações entre estudantes, tutores e professores\cite{Franciscato2008}. Um exemplo de um AVA utilizado em algumas instituições de ensino é o Moodle.
   
\section{O Moodle}
\section{Mapas conceituais e educação a distância}
% ---


% ---
% segundo capitulo de Resultados
% ---
\chapter{CMPAAS}
% ---
%O que é o CMPAAS?
%Quando foi criado?
%Porque foi criado? Qual é a sua importância?
% ---
\section{Pellentesque sit amet pede ac sem eleifend consectetuer}
% ---

\lipsum[24]

\chapter{Interoperabilidade de sistemas}
% ---
%O que é?
%Para que serve?
%Como é implementada?
%Json
% ---
\section{Pellentesque sit amet pede ac sem eleifend consectetuer}
% ---

\lipsum[24]

\chapter{Plugin}
% ---
%O que é um plugin do moodle?
%Como é feito o plugin?
%Como foi feito o plugin
% ---
\section{Pellentesque sit amet pede ac sem eleifend consectetuer}
% ---

\lipsum[24]

% ----------------------------------------------------------
% Finaliza a parte no bookmark do PDF
% para que se inicie o bookmark na raiz
% e adiciona espaço de parte no Sumário
% ----------------------------------------------------------
\phantompart

% ---
% Conclusão
% ---
\chapter{Conclusão}
% ---

\lipsum[31-33]

% ----------------------------------------------------------
% ELEMENTOS PÓS-TEXTUAIS
% ----------------------------------------------------------
\postextual
% ----------------------------------------------------------

% ----------------------------------------------------------
% Referências bibliográficas
% ----------------------------------------------------------
\bibliography{abntex2-modelo-references}

% ----------------------------------------------------------
% Glossário
% ----------------------------------------------------------
%
% Consulte o manual da classe abntex2 para orientações sobre o glossário.
%
%\glossary

% ----------------------------------------------------------
% Apêndices
% ----------------------------------------------------------

% ---
% Inicia os apêndices
% ---
%\begin{apendicesenv}

% Imprime uma página indicando o início dos apêndices
%\partapendices


%\end{apendicesenv}
% ---


% ----------------------------------------------------------
% Anexos
% ----------------------------------------------------------

% ---
% Inicia os anexos
% ---
%\begin{anexosenv}

% Imprime uma página indicando o início dos anexos
%\partanexos


%\end{anexosenv}

%---------------------------------------------------------------------
% INDICE REMISSIVO
%---------------------------------------------------------------------
%\phantompart
%\printindex
%---------------------------------------------------------------------

\end{document}
